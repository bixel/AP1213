\section{Aufbau und Durchf"uhrung}
	\label{sec:durchfuehrung}

	Die Solarzelle wird gem"a"s Abb. \eqref{aufbau} angeschlossen.
	Die vier Solarzellen werden per Br"uckenstecker auf einer Rastersteckplatte mit einem  Amperemeter und einer Widerstandsdekade in Reihe geschaltet, parallel dazu ein Voltmeter.

	\begin{figure}[htbp]
		\centering
		\includegraphics[width = 12cm]{img/aufbau.PNG}
		\caption{Schaltskizze zur Messung der I-U Kennlinie \cite{anleitung}}
		\label{aufbau}
	\end{figure}

	"Uber der Solarzelle wird eine $\SI{120}{\watt}$ Lampe an einer h"ohenverstellbaren Vorrichtung angebracht.
	Bei "uberbr"uckter Widerstandsdekade wird die H"ohe so eingestellt, dass der Kurzschlussstrom $I_K$ $\SI{30}{\milli\ampere}$, $\SI{50}{\milli\ampere}$, $\SI{75}{\milli\ampere}$ und $\SI{100}{\milli\ampere}$ betr"agt.
	Mit dem gr"o"sten Abstand wird die Messung begonnen.
	Nach Unterbrechung des Stromkreises wird f"ur jede Messreihe die Leerlaufspannung $U_0$ gemessen.\\
	Um die Strom-Spannungs-Kennlinie bei den einzelnen H"ohen zu bestimmen, wird nun die Br"ucke entfernt und die Widerstandsdekade von $\SI{1}{\ohm}$ bis $\SI{250}{\ohm}$ variiert.
	Dabei wird nach jeder Messung die Leistung mit Hilfe von $P = U \cdot I$ errechnet, um m"oglichst zeitnah die Messpunkte um das Leistungsmaximum durchzuf"uhren.
	Dabei ist darauf zu achten die Messreihen m"oglichst z"ugig aufzunehmen, da die Solarzelle sich aufheitzt.\\
	Aus den erhaltenen Daten wird nun der Wirkungsgrad bestimmt.