\section{Diskussion}
\label{sec:diskussion}
	Die durch diesen Versuch ermittelten Werte stimmen gut mit Werten aus Literatur "uberein.
	Bei quadratischen Stab aus Aluminium erwartet man ein Elastizit"atsmodul von ca $E = \SI{70}{\kilo \newton \per \milli \meter \squared}$ \cite{muenchen}.
	Der Gemessene Wert von $E = \SI{65.9(7)}{\kilo \newton \per \milli \meter \squared}$ entspricht dem relativ gut.
	Die Abweichung von etwa 7\% l"asst sich durch Messungenauigkeiten erkl"aren.

	Weil die Legierung des zweiten Stabes nicht bekannt ist, l"asst sich hier"uber nur schwer eine Aussage treffen.
	Der Wert der Elastizit"at $E$ liegt jedoch ebenfalls im Bereich des Aluminiums.

	Durch den Vergleich der Ergebnisse einseitiger und beidseitiger Einspannung l"asst sich jedoch sagen, dass die Methoden eine relativ gute "Ubereinstimmung der Messwerte haben.
	Somit wird das Ergebnis aus Kapitel \ref{subsec:einseitig} durch das Ergebnis aus Kapitel \ref{subsec:beidseitig} untermauert.
\begin{thebibliography}{9}

	\bibitem{anleitung} Physikalisches Anf"angerpraktikum der TU Dortmund: Versuch V103 - Biegung elastischer St"abe. \url{http://129.217.224.2/HOMEPAGE/PHYSIKER/BACHELOR/AP/SKRIPT/V103.pdf}. Stand: Juni 2013.

	\bibitem{muenchen} Uni Kiel. Elastizit"atsmodul in Zahlen. \url{http://www.tf.uni-kiel.de/matwis/amat/mw1_ge/kap_7/illustr/t7_1_2.html}. Stand: Juni 2013.
\end{thebibliography}