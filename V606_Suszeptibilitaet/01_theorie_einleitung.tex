\section{Einleitung} % (fold)
	\label{sec:einleitung}

	Alle Stoffe haben die Eigenschaft, Magnetfelder, die sie umgeben zu beeinflussen.
	Zun"achst wird das Feld geschw"acht, was als Diamagnetismus bezeichnet wird.
	Bei einigen Stoffen tritt ein verst"arkender Effekt ein, der den Diamagnetismus zum Teil weit "ubertrifft.
	Dies ist der hier behandelte Paramagnetismus.

	In Welcher Art das Magnetfeld ver"andert wird, wird dabei durch die Suszeptibilit"atskonstante $\chi$ beschrieben. Diese wird im Folgenden Versuch f"ur verschiedene stark paramagnetische seltene Erden untersucht.

\section{Theorie} % (fold)
	\label{sec:theorie}

	Die Magnetische Flussdichte $\vec{B}$ ist mit der magentischen Feldst"arke $\vec{H}$ "uber

	\begin{equation*}
		\vec{B} = \mu_0 \left(1 + \chi\right) \vec{H}
	\end{equation*}

	verkn"upft.
	Dabei ist $\chi$ keinesfalls konstant, sondern h"angt von der Temperatur $T$ und der Beschaffenheit des Feldes $\vec{H}$ ab.

	Ein Atom, Ion oder Molek"ul mit nicht verschwindendem Drehimpuls ist in der Lage, sich an einem "au"seren Magnetfeld auszurichten.
	Der Gesamtdrehimpuls $\vec{J}$ setzt sich dabei aus den Anteilen des Bahndrehimpulses $\vec{L}$, des Gesamtspins $\vec{S}$ und des zu vernachl"assigenden Kerndrehimpulses ab:

	\begin{equation*}
		\vec{J} = \vec{L} + \vec{S}\,.
	\end{equation*}

	Zu den Drehimpulsen $\vec{L}$ und $\vec{S}$ geh"oren, entsprechend der Quantenmechanik, die magnetischen Momente $\vec{\mu}_\mathrm{L}$ und $\vec{\mu}_\mathrm{S}$.
	Nach einiger Rechnung f"uhren diese Gr"o"sen auf den gen"aherten Betrag des gesamten Magnetischen Moments:

	\begin{equation*}
		\vec{\mu}_\mathrm{J} = \mu_\mathrm{B} g_\mathrm{J} \sqrt{J(J + 1)}\,.
	\end{equation*}

	Diese F"ahigkeit wird durch die Temperatur $T$ beeinflusst, weil die Ausrichtung durch thermische Bewegungen gest"ort wird.
	Das Curiesche Gesetz des Paramagnetismus beschriebt diesen Zusammenhang mit

	\begin{equation*}
		\chi \propto \frac{1}{T}\,.
	\end{equation*}

	Bei der Untersuchung der Suszeptibilit"at ist also darauf zu achten, dass Temperaturschwankungen in den Stoffen vermieden werden.

	\subsection{Berechnung der Suszeptibilit"at seltener Erden}
		\label{subsec:berechnung}


