\section{Auswertung}
	\label{sec:auswertung}

	\subsection{Filterkurve des Selektivverst"arkers} % (fold)
	\label{sub:subsection_name}
	
	Die Untersuchung der Filterkurve des Selektiv-Verst"arkers ergab die in Tabelle \ref{tabelle:aufgabe_a} aufgelisteten Werte.
	Daraus ergibt sich der Graph \ref{graph:aufgabe_a}.
	Aus diesem lassen sich die Werte f"ur $\nu_\mathrm{0}$, $\nu_\mathrm{+}$ und $\nu_\mathrm{-}$ ablesen:

	\begin{eqnarray*}
		\nu_\mathrm{0} &=& \SI{34.9}{\kilo\hertz} \\
		\nu_\mathrm{+} &=& \SI{35.10 (2)}{\kilo\hertz}\\
		\nu_\mathrm{-} &=& \SI{34.72 (2)}{\kilo\hertz}
	\end{eqnarray*}

	Aus diesen l"asst sich die G"ute mithilfe von Gleichung \eqref{Gleichung:Guete} berechnen.

	\begin{equation*}
		Q = \SI{91.84 (684)}{}
	\end{equation*}

	Der Fehler ergibt sich mittels Gau"s'scher Fehlerfortpflanzung:

	\begin{eqnarray*}
		|\frac{\partial Q}{\partial\nu_\mathrm{+}}| &=& |\frac{\partial Q}{\partial\nu_\mathrm{-}}| \\
		\Delta \nu_\mathrm{+} &=& \Delta \nu_\mathrm{-} \\
		\Delta Q &=& \sqrt{ 2 \cdot \left( |\frac{\partial Q}{\partial\nu_\mathrm{+}}| \Delta \nu_\mathrm{+,-} \right)^2}
	\end{eqnarray*}

	Damit weicht die errechnete G"ute Q = $\SI{91.84 (684)}{}$ um etwa $\SI{8.8}{\%}$ von der eingestellten G"ute Q = $100$ ab.

	\begin{table}[!h]
\begin{center}
\begin{tabular}{|r|r|r|r|}
\hline
t[$\SI{}{\micro\second}$] & |U[$\SI{}{\volt}$]| & t[$\SI{}{\micro\second}$] & |U[$\SI{}{\volt}$]| \\
\hline
\hline
0	&	175	& 199	&	60\\
20	&	165	& 212	&	55\\
38	&	149	& 229	&	51\\
50	&	140	& 241	&	45\\
68	&	122	& 258	&	43\\
80	&	115	& 272	&	40\\
98	&	105	& 289	&	39\\
110 &	99	& 301	&	35\\
125 &	90	& 318	&	31\\
140 &	85	& 330	&	27\\
155 &	75	& 345	&	26\\
170 &	70	& 360	&	25\\
183 &	65	& 375	&	24\\
\hline
\end{tabular}
\caption[Messwerte zu a]{Messwerte aus dem Graphen \eqref{amplitude} angenommen}
\label{werte_a}
\end{center}
\end{table}

	\begin{figure}[!h]
		\centering
		\includegraphics[width = 14cm]{img/arschlecken.pdf}
		\caption{Filterkurve des Selektivverst"arkers}
		\label{graph:aufgabe_a}
	\end{figure}