\documentclass{scrartcl}
	%ngerman fur Umlaute, Anf"hrungszeichen, etc.%
	\usepackage{ngerman}

	%siunitx fur SI-Einheitesnsystem%
	\usepackage{siunitx}
	\sisetup{
	    locale=DE,
	    separate-uncertainty=true,
    	per-mode=fraction
	}

	%graphicx fur Bildeinbindung%
	\usepackage{graphicx}

	%Texteinzug vor Absatz entfernen%
	\parindent0pt

\begin{document}
	
	\section{Theorie}

		\subsection{Ableitung einer allgemeinen Relaxationsgleichung und ihre Anwendung auf den RC-Kreis}

			Wenn ein System aus seinem Ausgangszustand entfernt wird und es wieder nicht-oszillatorisch in denselben zurückkehrt, treten Relaxationserscheinungen auf. Die "Anderungsgeschwindigkeit zum Zeitpunkt $t$ der physikalischen Gr"o"se $A$ ist dabei in den meisten F"allen proportional zur Abweichung der Gr"o"se $A$ vom Endzustand $A(\infty)$. Dieser ist nur asymptotisch errechenbar.

			\begin{equation}
				\frac{\mathrm{d} A}{\mathrm{d} t} = c \left[ A(t) - A(\infty) \right] \label{1}
			\end{equation}

			durch Integration vom Zeitpunkt 0 bis t liefert:

			\begin{equation}
				\int_{A(0)}^{A(t)} \! \frac{\mathrm{d} A'}{A' - A(\infty)} = \int_0^t \! c \, \mathrm{d} t' \label{2}
			\end{equation}

			oder

			\begin{equation}
				\ln{\frac{A(t) - A(\infty)}{A(0) - A(\infty)}} = c t \label{3}
			\end{equation}

			und daraus folgt:

			\begin{equation}
				A(t) = A(\infty) + \left[ A(0) - A(\infty) \right] \, e^{c t} \label{4}
			\end{equation}

			wobei in \ref{4} $c>0$ sein muss, damit $A$ beschr"ankt bleibt.

			\subsubsection{Relaxationsvorg"ange am Beispiel eines Kondensators}

				Der Auf- und Entladevorgang eines Kondensators "uber einen Widerstand stellt einen Relaxationsvorgang da.

				Entladevorgang:

				Wenn auf den Platten eines Kondensators mit der Kapazität $C$ die Ladung $Q$ liegt, so liegt zwischen ihnen die Spannung $U_\mathrm{C}$:

				\begin{equation}
					U_\mathrm{C} = \frac{Q}{C} \label{5}
				\end{equation}

				Nach dem ohmschen Gesetz f"uhrt bedingt diese einen Strom durch den Widerstand $R$:

				\begin{equation}
					I = \frac{U_\mathrm{C}}{R} \label{6}
				\end{equation}

				Da auf dem Zeitintervall $\mathrm{d}t$ die Ladung $-\mathrm{d} I$ flie"st, "andert sich die Ladung auf dem Kondensator um: 

				\begin{equation}
					\mathrm{d} Q = -\mathrm{d}I \label{7}
				\end{equation}

				Mit Hilfe der GLeichungen \ref{5}, \ref{6} und \ref{7} kann $U_\mathrm{C}$ und $I$ eliminiert werden und eine Differentialgleichung entsteht.

				\begin{equation}
					\frac{\mathrm{d}Q}{\mathrm{d}t} = -\frac{1}{RC}Q(t)
				\end{equation}

				mit der Anfangsbedingung

				\begin{equation}
					U(\infty) = 0
				\end{equation}

				Daraus liefert die Integration aus \ref{2}

				\begin{equation}
					Q(t) = Q(0) \exp{ \left( -\frac{t}{RC} \right) }
				\end{equation}

				Aufladevorgang:

				Wie beim Entladevorgang l"asst sich der Aufladevorgang berechnen, doch werden hier die Anfangsbedingungen

				\begin{equation}
					Q(0) = 0 \, \, \, \mathrm{und} \, \, \, Q(\infty) = C U_\mathrm{0}
				\end{equation}

				genutzt, wodurch sich f"ur den Aufladevorgang ergibt

				\begin{equation}
					Q(t) = C U_\mathrm{0} \left( 1 - \exp{ \left( -\frac{t}{RC} \right) } \right)
				\end{equation}

				wobei $RC$ als Zeitkonstante bezeichnet wird und ist ein Ma"s f"ur die Geschwindigkeit des jeweiligen Vorgangs. W"ahrend des Zeitraums 
				$\Delta T = RC$ "andert sich die Ladung des Kondensators um

				\begin{equation}
					\frac{Q(t = RC)}{Q(0)} = \frac{1}{e} (ungef"ahr) 0,368
				\end{equation}

				$=>$ Nach $\Delta T = 2,3 \, RC$ sind noch $10 \%$ des Ausgangswerts vorhanden und nach $\Delta T = 4,6 \, RC$ noch etwa $1 \%$.

		\subsection{Relaxationsph"anomene, die unter dem Einfluss einer periodischen Auslenkung aus der Gleichgewichtslage auftreten}

			Im Folgenden wird das wiederum das Verhalten des RC-Kreises, an welchem eine Sinusspannung anliegt, betrachtet, da dies eine enge Analogie zu einem mechanischen System besitzt, welches unter dem Einfluss einer Kraft mit sinusf"ormiger Zeitabh"angigkeit steht.


			Wenn f"ur die Kreisfrequenz $\omega$ der "au"seren Wechselspannung $U(t)$ mit
			\begin{equation}
				U(t) = U_\mathrm{0} \cos{\omega t}
			\end{equation}

	\section{Auswertung}

		\subsection{Messaufgaben}

			\begin{enumerate}
				\item Man bestimme die Zeitkonstante eines RC-Gliedes durch Beobachtung des Auf oder Entladevorganges des Kondensators.
				\item Man messe die Amplitude der Kondensatorspannung an einem RC-Glied, welches an einem Sinusspannungsgenerator angeschlossen ist, in Abhängigkeit von der Frequenz.
				\item Man messe die Phasenverschiebung zwischen Generator- und Kondensatorspannung an einem RC-Glied in Abhängigkeit von der Frequenz.
				\item Man zeige, dass ein RC-Kreis unter bestimmten Voraussetzungen als Integrator arbeiten kann.
			\end{enumerate}
\end{document}
