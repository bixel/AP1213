\documentclass{scrartcl}
	%ngerman fur Umlaute, Anf"hrungszeichen, etc.%
	\usepackage{ngerman}

	%siunitx fur SI-Einheitesnsystem%
	\usepackage{siunitx}
	\sisetup{
	    locale=DE,
	    separate-uncertainty=true,
    	per-mode=fraction
	}

	%graphicx fur Bildeinbindung%
	\usepackage{graphicx}

	%Texteinzug vor Absatz entfernen%
	\parindent0pt

\begin{document}

	\section{Auswertung}

		Zun"achst sollte die Zeitkonstante $\lambda$ durch die Beobachtung eines Auf- oder Ent\-lade\-vor\-ganges des Kondensators bestimmt werden.
		Anschlie"send wurden Phasenunterschied $\phi$ und Amplitude $U_{\mathrm{C}}$ in Abh"angigkeit der Frequenz $\omega$ aufgezeichnet, um die in \ref{theorie} hergeleiteten Beziehungen zu untersuchen.
		Die Funktion des RC-Gliedes als Integrierer haben wir schließlich mit einigen Screenshots des Oszilloskopes veranschaulicht (siehe *009* bis *014*\ref{abb_ss_phase}).

		\subsection{Bestimmung der Zeitkonstante $\lambda$}

			Hierf"ur haben wir einen Entladevorgang des Kondensators betrachtet und mit Hilfe des Oszilloskopes
			zehn Messwerte der Spannung $U_{\mathrm{C}}$ zu verschiedenen Zeitpunkten $t$ genommen.\\

			* 004 *\\

			Indem wir die Werte auf halblogarithmischem Papier einzeichnen, l"asst sich $\lambda$ leicht als Steigung der Ausgleichsgeraden durch die Werte berechnen.
			Tabelle \ref{tabelle_a1} zeigt die Messdaten.
			Eine lineare Regression mit numpy ergibt eine Steigung von $m = $ und ein Offset von $b = $.
			Abbildung \ref{graph_t_uc} zeigt den entsprechenden Graphen. Der Funktionsgenerator lieferte hierbei eine rechteckf"ormige Spannung $U_0$ der Frequenz $\omega = \SI{55}{\hertz}$.\\

			* Tabelle zu Aufgabe 1 * \\

			* Graph t UC * \\

			Anmerkung: Die Messfehler waren zu klein, um sie h"andisch einzeichnen zu k"onnen.\\

			Die Daten der Ausgleichsrechnung liefern f"ur

			\begin{displaymath}
				R_{ges}C = \SI{1.3476 (136)}{\milli\second}
			\end{displaymath}

		\subsection{Frequenzabh"angigkeit der Amplitude $U_{\mathrm{C}}$}

			Hier sollte der Zusammenhang \ref{theorie_amplitude_frequenz} "uberpr"uft werden.
			Daf"ur haben wir im Versuchsaufbau \ref{abb_aufbau_2} mit dem Millivoltmeter 17 Messwerte der Spannung $U_{\mathrm{C}}$ aufgenommen.
			Weil $U_{\mathrm{C}}$ im Bereich von einigen Hertz leicht schwankte,
			mitteln wir dort "uber f"unf bzw. sechs Werte.
			Tabelle \ref{tabelle_a2} zeigt die Messwerte.\\

			* Tabelle zu Aufgabe 2 * \\

			Im folgenden Graphen erkennt man gut, dass das RC-Glied bei gro"sen Frequenzen $\omega$ sperrt.\\

			* Graph zu Aufgabe 2 * \\


		\subsection{Frequenzabh"angigkeit der Phasendifferenz $\phi$}

			Nun untersuchen wir die Beziehung \ref{theorie_phase_frequenz} aus Teil \ref{theorie}.
			Nachdem die Nulldurchg"ange der Konden\-sator\-spannung $U_{\mathrm{C}}$ und der Eingangsspannung $U_0$ angeglichen wurden,
			lie"s sich die Phasendifferenz $\phi$, wie in \ref{durchf} beschrieben, bestimmen.
			Die Werte wurden mit dem Curser des digitalen Oszilloskopes aufgenommen.

			Diese Messung wurde f"ur Frequenzen von $\SI{1}{\hertz}$ bis $\SI{1}{\kilo\hertz}$ durchgef"uhrt. Tabelle \ref{tabelle_a3} zeigt die Messdaten, sowie die resultierenden Werte f"ur $a$ und $b$.

			* Tabelle zu Aufgabe 3 * \\

			* Graph zu Aufgabe 3 * \\

			Der Vergleich mit der Theoriekurve zeigt, dass die Vorhersage gut erf"ullt wird.

		\subsection{RC-Glied als Integrierglied}

			Zuletzt haben wir eine Frequenz von $\omega = \SI{10}{\kilo\hertz}$ gew"ahlt und damit $\omega \gg \frac{1}{RC}$ gew"ahlt.
			Die folgenden Abbildungen zeigen das Eingangssignal $U_0$ und das Ausgangssignal $U_{\mathrm{C}}$.\\

			* Abbildung A3.1 * \\

			* Abbildung A3.2 * \\

			* Abbildung A3.3 * \\

			* Abbildung A3.4 * \\

\end{document}
