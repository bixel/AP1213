\documentclass{scrartcl}
	%ngerman fur Umlaute, Anf"hrungszeichen, etc.%
	\usepackage{ngerman}

	%siunitx fur SI-Einheitesnsystem%
	\usepackage{siunitx}
	\sisetup{
	    locale=DE,
	    separate-uncertainty=true,
    	per-mode=fraction
	}

	%graphicx fur Bildeinbindung%
	\usepackage{graphicx}

	%Texteinzug vor Absatz entfernen%
	\parindent0pt

\begin{document}
	
	\section{Theorie}

		\subsection{Ableitung einer allgemeinen Relaxationsgleichung und ihre Anwendung auf den RC-Kreis}

			Wenn ein System aus seinem Ausgangszustand entfernt wird und es wieder nicht-oszillatorisch in denselben zurückkehrt, treten Relaxationserscheinungen auf. Die "Anderungsgeschwindigkeit zum Zeitpunkt $t$ der physikalischen Gr"o"se $A$ ist dabei in den meisten F"allen proportional zur Abweichung der Gr"o"se $A$ vom Endzustand $A(\infty)$. Dieser ist nur asymptotisch errechenbar.

			\begin{equation}
				\frac{\mathrm{d} A}{\mathrm{d} t} = c \left[ A(t) - A(\infty) \right]
			\end{equation}

			durch Integration vom Zeitpunkt 0 bis t liefert:

			\begin{equation}
				\int_{A(0)}^{A(t)} \! \frac{\mathrm{d} A'}{A' - A(\infty)} = \int_0^t \! c \, \mathrm{d} t'
			\end{equation}

			oder

			\begin{equation}
				\ln{\frac{A(t) - A(\infty)}{A(0) - A(\infty)}} = c t
			\end{equation}

			und daraus folgt:

			\begin{equation}
				A(t) = A(\infty) + \left[ A(0) - A(\infty) \right] \, e^{c t} 
			\end{equation}


	\section{Auswertung}

		\subsection{Messaufgaben}

			\begin{enumerate}
				\item Man bestimme die Zeitkonstante eines RC-Gliedes durch Beobachtung des Auf oder Entladevorganges des Kondensators.
				\item Man messe die Amplitude der Kondensatorspannung an einem RC-Glied, welches an einem Sinusspannungsgenerator angeschlossen ist, in Abhängigkeit von der Frequenz.
				\item Man messe die Phasenverschiebung zwischen Generator- und Kondensatorspannung an einem RC-Glied in Abhängigkeit von der Frequenz.
				\item Man zeige, dass ein RC-Kreis unter bestimmten Voraussetzungen als Integrator arbeiten kann.
			\end{enumerate}
\end{document}
