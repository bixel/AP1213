
\begin{table}[!h]
\begin{center}
\begin{tabular}{|r|r|r|r|r|r|}
\hline
t[$\SI{}{\minute}$] & $T_\mathrm{1}[\SI{}{\kelvin}$] & $T_\mathrm{2}[\SI{}{\kelvin}$] & $p_\mathrm{a}[\SI{}{\bar}]$ & $p_\mathrm{b}[\SI{}{\bar}]$ & A[$\SI{}{\watt}$]\\
\hline
\hline

 1 &	23,7 &	19,6 &	2,0 &	 6,50 &	200\\
 2 &	25,9 &	17,6 &	2,1 &	 7,00 &	200\\
 3 &	28,3 &	15,4 &	2,2 &	 7,25 &	205\\
 4 &	30,6 &	13,4 &	2,2 &	 7,75 &	210\\
 5 &	32,9 &	11,2 &	2,2 &	 8,25 &	210\\
 6 &	35,1 &	 9,4 &	2,2 &	 8,75 &	210\\
 7 &	37,3 &	 7,5 &	2,2 &	 9,00 &	215\\
 8 &	39,3 &	 5,7 &	2,2 &	 9,50 &	215\\
 9 &	41,2 &	 4,0 &	2,2 &	10,00 &	215\\
10 &	43,1 &	 2,5 &	2,2 &	10,50 &	215\\
11 &	44,9 &	 1,4 &	2,2 &	11,00 &	215\\
12 &	46,7 &	 0,7 &	2,2 &	11,50 &	215\\
13 &	48,3 &	 0,1 &	2,2 &	12,00 &	215\\
14 &	49,8 &	-0,3 &	2,2 &	12,50 &	215\\

\hline
\end{tabular}
\caption[]{Aufgenommene Messgr"o"sen zur Bestimmung der G"uteziffer der Apparatur, dem Massendurchsatz des Transportgases und der mechanischen Leistung des Kompressors.}
\label{daten}
\end{center}
\end{table}