\section{Einleitung} % (fold)
\label{sec:einleitung}
	Die Ultraschalltechnik findet eine weit verbreitete Anwendung.
	Sie wird h"aufig im medizinischen Bereich zur Diagnostik, sowie im technischen Bereich zur Materialanalyse benutzt.
	Die grundlegende Funktionsweise dieser Technik soll hier n"aher untersucht werden.
\section{Theorie} % (fold)
\label{sec:theorie}
	Schall bezeichnet im Allgemeinen Druckver"anderungen in einem Medium.
	In Luft und Fl"ussigkeiten breiten sich diese Druckver"anderungen als Longitudinalwellen aus.
	Die Frequenz $\nu$ dieser Wellen wird in vier Bereiche unterteilt.
	Der menschlichen H"orbereich liegt in etwa bei $\nu = \SI{16}{\hertz}$ bis $\nu = \SI{20}{\kilo \hertz}$.
	Bis zu einer Frequenz von etwa $\nu = \SI{1}{\giga \hertz}$ werden Schallwellen als Ultraschall bezeichnet.
	Bei noch h"oheren Frequenzen handelt es sich um Hyperschall.
	Der Bereich unter $\nu = \SI{16}{\hertz}$ wird als Infraschall bezeichnet.

	Eine eindimensionale Schallwelle $p$, die sich in $x$-Richtung ausbreitet wird beschrieben durch

	\begin{equation}
		p(x,t) = p_0 + v_0 Z \cos{(\omega t - kx)} \,.
	\end{equation}

	Dabei bezeichnet $p_0$ den Normaldruck, $v_0$ die Schallschnelle und $Z = \rho c$ die akustische Impedanz mit der Dichte $\rho$ und der Schallgeschwindigkeit $c$.
