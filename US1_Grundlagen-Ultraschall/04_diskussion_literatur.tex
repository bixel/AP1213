\section{Diskussion}
\label{sec:diskussion}
	
	Der Versuch half bei dem Verst"andnis der Wirkungsweise eines Ultraschalger"ates. Die Ergebnisse f"ur die Schallgeschwindigkeiten stimmten in guter N"aherung an Literaturwerten von $\approx \SI{2.7}{\milli\meter\per\micro\per\second}$ \cite{glas} "uberein.

	Bei der Abmessung der Lochbreiten von den St"orstellen sind gro"se Unterschiede zwischen den verschiedenen Frequenzen der Sonden erkennbar.

	So sind hochfrequentige Sonden f"ur den Nahbereich geeignet, da diese dort sehr gut aufl"osen k"onnen, w"ahrend niederfrequente Sonden eher f"ur den Tiefenscan geeignet sind, wobei diese auch dort keine hohen Aufl"osungen besitzen.
	So waren die Maxima breit und hatten teilweise keine klare Spitze, bzw. beim B-Scan gibt es wenige bis keine roten/scharfen Stellen.

	Die Vermessung des Auges mittels Ultraschall hat gut funktioniert.

	Abschlie"send l"asst sich sagen, dass nicht f"ur jedes Problem dieselbe Sonde sinnvoll sein kann, sondern dass je nach Anwendung verschiedene Sonden benutzt werden sollten um ein bestm"ogliches Ergebnis zu erreichen. 

\begin{thebibliography}{9}
	\bibitem{anleitung} Physikalisches Anf"angerpraktikum der TU Dortmund: Versuch US1 - Grundlagen der Ultraschalltechnik. \url{http://129.217.224.2/HOMEPAGE/PHYSIKER/BACHELOR/AP/SKRIPT/UltraschallGL.pdf}. Stand: Juni 2013.

	\bibitem{glas} Schallgeschwindigkeit in Acrylglas. \url{http://www.plexiglas.de/product/plexiglas/de/ueber/glossar/pages/default.aspx}.
\end{thebibliography}


