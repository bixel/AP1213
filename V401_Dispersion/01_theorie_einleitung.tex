\section{Einleitung} % (fold)
\label{sec:einleitung}

	Durch die Wechselwirkung einer Lichtwelle mit den Elektronen der Materie werden die Ausbreitungsgeschwindigkeiten $v$ des Lichtes kleiner als die Vakuumlichtgeschwindigkeit $c$.
	Es wird sich herausstellen, dass die Geschwindigkeit von der Wellenl"ange des Lichtes abh"angt. Dieses Ph"anomen wird als Dispersion bezeichnet.
	
\section{Theorie} % (fold)
\label{sec:theorie}

	\subsection{Brechung, Brechungsindex und Dispersionskurve} % (fold)
	\label{sub:brechungsindex}
	
	Tritt ein Lichtstrahl schr"ag in Materie ein, so erf"ahrt er durch die "Anderung der Geschwindigkeit an der Grenzfl"ache eine Richtungs"anderung.

	Dies wird als Brechung bezeichnet und wird durch den Brechungsindex $n$ beschrieben.

	Dieser ist definiert durch das Verh"altnis der beiden Lichtgeschwindigkeiten.

	\begin{equation}
		n := \frac{v_\mathrm{1}}{v_\mathrm{2}} \qquad . \label{gl:brechung}
	\end{equation}

	Daraus folgt, dass auch der Brechungsindex $n$ eine frequenzabh"angige bzw. Wellenl"angenabh"angige Gr"o"se ist und somit eine Funktion im Bereich des sichtbaren Lichtes.

	Dies wird als Dispersionskurve bezeichnet.

	\begin{equation}
		n = f(\lambda) \qquad .
	\end{equation}

	\subsection{Das Huygenssche Prinzip} % (fold)
	\label{sub:das_huygenssche_prinzip}
	
	\begin{figure}[!h]
		\centering
		\includegraphics[width = 6cm]{img/huygen.jpg}
		\caption{Das Huygenssche Prinzip und die daraus resultierenden wichtigen Gr"o"sen f"ur die Herleitung des Snelliusschen Brechungsgesetzes \cite{anleitung}.}
		\label{huygen}
	\end{figure}

	Jeder Punkt einer bestehenden Wellenfl"ache kann als Zentrum einer neuen kugelf"ormigen "`Elementarwelle"' aufgefasst werden. Die Einh"ullende aller Elementarwellen gibt die Wellenfront f"ur einen sp"ateren Zeitpunkt. Durch die Geschwindigkeits"anderung an einer Grenzfl"ache kommt so eine Richtungs"anderung zustande (Abb. \ref{huygen}).

	Aus Abb. \ref{huygen} l"asst sich die Beziehung \eqref{gl:huygen} ablesen.

	\begin{equation}
		\frac{\sin(\alpha)}{\sin(\beta)} = \frac{v_\mathrm{1}}{v_\mathrm{2}} \qquad . \label{gl:huygen}
	\end{equation}

	Aus Gleichung \eqref{gl:brechung} ergibt sich damit das Snelliussche Brechungsgesetz

	\begin{equation}
		\frac{\sin(\alpha)}{\sin(\beta)} = n \qquad .
	\end{equation}