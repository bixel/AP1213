\section{Auswertung}
	\label{sec:auswertung}
	Vor allen Messungen musste der Nulleffekt ermittelt werden, um unsere Messwerte sp"ater von diesem zu bereinigen.
	Hierf"ur wurden die Zerf"alle $N_{\mathrm{u}}$, die ohne Probe ermittelt werden, "uber lange Zeit gez"ahlt.
	Anschlie"send konnten die eigentlichen Messungen durch\-ge\-f"uhrt werden.
	

	Um auf die Halbwertszeiten zu schlie"sen konnte die Beziehung \eqref{eqn:t_halb} aus Kapitel \ref{sec:theorie} ver\-wen\-det werden.
	Weil die Messwerte Poisson-verteilt sind, betrug der Fehler auf die Messwerte

	\begin{eqnarray*}
		\Delta N_{\Delta t, \mathrm{gem}} & = & \sqrt{N_{\Delta t, \mathrm{gem}}}, \\
		\Delta N_{\Delta t, \mathrm{u}} & = & \sqrt{N_{\Delta t, \mathrm{u}}},\\
		\Rightarrow \Delta N_{\Delta t} & = & \sqrt{N_{\Delta t, \mathrm{gem}} + N_{\Delta t, \mathrm{u}}} .
	\end{eqnarray*}

	$N_{\Delta t, \mathrm{u}}$ bezeichnet dabei den Nulleffekt pro Messung. Dieser betrug f"ur das Messintervall $\Delta t$

	\begin{equation*}
		\Delta N_{\Delta t, \mathrm{u}} = \frac{256}{900} \SI{}{\per \second} \cdot \Delta t .
	\end{equation*}

	Um eine Ausgleichsrechnung durch\-zu\-f"uh\-ren wurden die Werte linearisiert.
	Danach konnte eine lineare Regression mit scipy durchgef"uhrt werden, welche $\lambda$ und somit die Halb\-werts\-zei\-ten $T$ lieferte.

	\subsection{Halbwertszeit von Indium}
		\label{subsec:indium}
		Es wurden 15 Messungen "uber jeweils $\Delta t = \SI{240}{\second}$ durchgef"uhrt.
		Anschlie"send konnte, wie oben beschrieben, durch lineare Regression $T_\mathrm{In}$ ermittelt werden. Es musste dabei wiederum ein Gau"s'scher Fehler ber"ucksichtigt werden:

		\begin{displaymath}
			\Delta T = \left|\frac{\partial T}{\partial \lambda} \cdot \Delta \lambda \right| = \frac{\Delta \lambda}{\lambda^2} \ln{2} ,
		\end{displaymath}

		wobei $\Delta \lambda$ den Fehler der linearen Regression bezeichnet.
		Es ergaben sich die Werte

		\begin{eqnarray*}
			\lambda & = & \SI{188.9 (10)e-6}{\per \second} , \\
			T_\mathrm{In} & = & \SI{3668 (186)}{\second} .
		\end{eqnarray*}

		\begin{table}[!h]
			\begin{center}
				\label{tabelle:indium}
				\caption{Messwerte f"ur Indium-116}
				\begin{tabular}{|c|c|c||c|c|c|}
					\hline 
					Messung & $N_{\Delta t, \mathrm{gem}}$ & $N_{\Delta t}$ & Messung & $N_{\Delta t, \mathrm{gem}}$ & $N_{\Delta t}$\\
					\hline 
					\hline
					1 & 2265 & 2197 & 9 & 1696 & 1628\\
2 & 2236 & 2168 & 10 & 1613 & 1545 \\
3 & 2243 & 2175 & 11 & 1519 & 1451 \\
4 & 2038 & 1970 & 12 & 1421 & 1353 \\
5 & 1926 & 1858 & 13 & 1412 & 1344 \\
6 & 1875 & 1807 & 14 & 1352 & 1284 \\
7 & 1639 & 1571 & 15 & 1203 & 1135 \\
8 & 1756 & 1688 & & & \\
					\hline 
				\end{tabular}
			\end{center}
		\end{table}

		\begin{figure}[!h]
			\centering
			\includegraphics[width = 15cm]{img/graph_indium_linearisiert.pdf}
			\caption{Bereinigte Messwerte und Graph f"ur Indium-116}
			\label{fig:indium}
		\end{figure}

	\clearpage

	\subsection{Halbwertszeit von Rhodium}
		\label{subsec:rhodium}
		Bei dieser Messung war zu beachten, dass die Probe zwei verschiedene Isotope ${}_{45}^{104}\mathrm{Rh}$ und das angeregte ${}_{45}^{104i}\mathrm{Rh}$ enthielt.
		Weil beide Isotope aber unterschiedliche Halbwertszeiten $T_\mathrm{Rh}$ haben,
		konnte zun"achst, wie in Kapitel \ref{sec:theorie} beschrieben, die gr"o"sere Halbwertszeit $T_{\mathrm{Rh}, 2}$ bestimmt werden.

		Anschlie"send wurden die hierdurch verursachten Zerf"alle von den ersten 18 Messwerten abgezogen und wie in Kapitel \ref{subsec:indium} fortgefahren.

		Hierbei wurde "uber jeweils $\Delta t = \SI{12}{\second}$ gemessen.
		Die Zeit $t^*$ wurde zu $\SI{312}{\second}$ gew"ahlt.
		$T_{\mathrm{Rh}, 1}$ bezeichnet im Folgenden die Halbwertszeit f"ur die ersten 18 Messwerte, also ${}_{45}^{104}\mathrm{Rh}$ und $T_{\mathrm{Rh}, 2}$ f"ur die letzten 14 Messwerte, also ${}_{45}^{104i}\mathrm{Rh}$.
		\newline


		F"ur ${}_{45}^{104}\mathrm{Rh}$ ergab sich dann

		\begin{eqnarray*}
			\lambda & = & \SI{16.04 (86)e-3}{\per \second} , \\
			T_\mathrm{In} & = & \SI{43.2 (233)}{\second} .
		\end{eqnarray*}

		F"ur ${}_{45}^{104i}\mathrm{Rh}$ lieferten die Messwerte

		\begin{eqnarray*}
			\lambda & = & \SI{2.74 (18)e-3}{\per \second} , \\
			T_\mathrm{In} & = & \SI{253.2 (1683)}{\second} .
		\end{eqnarray*}

		\clearpage

		\begin{table}[!h]
			\begin{center}
				\caption{Messwerte f"ur Rhodium}
				\label{tabelle:rhodium}
				\begin{tabular}{|c|c|c||c|c|c||c|c|c|}
					\hline 
					Messung & $N_{\Delta t, \mathrm{gem}}$ & $N_{\Delta t}$ & Messung & $N_{\Delta t, \mathrm{gem}}$ & $N_{\Delta t}$ & Messung & $N_{\Delta t, \mathrm{gem}}$ & $N_{\Delta t}$\\
					\hline 
					\hline
					1	&	449	&	446		&	15	&	69	&	66	&	29	&	23 & 20	\\
2	&	405	&	402		&	16	&	72	&	69	&	30	&	37 & 34	\\
3	&	402	&	399		&	17	&	48	&	45	&	31	&	25 & 22	\\
4	&	356	&	353		&	18	&	55	&	52	&	32	&	24 & 21	\\
5	&	278	&	275		&	19	&	51	&	48	&	33	&	17 & 14	\\
6	&	247	&	244		&	20	&	50	&	47	&	34	&	19 & 16	\\
7	&	165	&	162		&	21	&	40	&	37	&	35	&	19 & 16	\\
8	&	155	&	152		&	22	&	30	&	27	&	36	&	23 & 20	\\
9	&	162	&	159		&	23	&	41	&	38	&	37	&	30 & 27	\\
10	&	142	&	139		&	24	&	31	&	28	&	38	&	23 & 20	\\
11	&	112	&	109		&	25	&	46	&	43	&	39	&	21 & 18	\\
12	&	116	&	113		&	26	&	23	&	20	&	40	&	25 & 22	\\
13	&	97	&	94		&	27	&	27	&	24	&		&	&\\	
14	&	83	&	80		&	28	&	43	&	40	&		&	&\\
					\hline 
				\end{tabular}
			\end{center}
		\end{table}

		\enlargethispage{5cm}

		\begin{figure}[!h]
			\centering
			\includegraphics[width = 14.2cm]{img/graph_rhodium_linearisiert.pdf}
			\caption{Bereinigte Messwerte und Graph f"ur Rhodium-104 und Rhodium 104i}
			\label{fig:the_label}
		\end{figure}

		\clearpage
