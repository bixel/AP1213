\newpage
\section{Diskussion}
	\label{sec:diskussion}
	Dieser Versuch stellt eine M"oglichkeit zur Bestimmung der Viskosit"at von Wasser vor.
	Alle damit ermittelten Werte liegen weit "uber den Literaturwerten \cite{uni_magdeburg}.
	Bei $T = \SI{20}{\celsius}$ lag die Abweichung bei $\SI{19}{\percent}$.
	Mit steigender Temperatur vergr"o"serte sich der Unterschied sogar noch und stieg bei $T = \SI{54}{\celsius}$ auf etwa $\SI{22}{\percent}$ an.
	Diese Unterschiede lassen sich nur duch systematische Fehler erkl"aren.
	Die Messung liefert also insgesamt eher schlechte Werte.	

\begin{thebibliography}{9}
	\bibitem{anleitung} Physikalisches Anf"angerpraktikum der TU Dortmund: Versuch Nr. 107 - Das Kugelfallviskosimeter nach H"oppler. Stand: November 2012.

	\bibitem{uni_magdeburg} Universit"at Magdeburg, Institut f"ur Str"omungsmechanik und Thermodynamik: Stoffwerte von Wasser. www.uni-magdeburg.de/isut/LSS/Lehre/Arbeitsheft/IV.pdf. Stand: 2. Dezember 2012.

	\bibitem{walcher} Walcher, W.: Praktikum der Physik. Teubner Studienb"ucher, Teubner-Verlag. Stuttgart.
\end{thebibliography}