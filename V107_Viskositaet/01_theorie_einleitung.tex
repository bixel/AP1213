\section{Einleitung}
	\label{sec:einleitung}

	Die dynamische Viskosit"at von Fl"ussigkeiten ist temperaturabh"angig. In diesem Versuch soll die dynamische Viskosit"at von destilliertem Wasser mit Hilfe des Kugelfall-Vis\-ko\-si\-me\-ters nach H"oppler bestimmt werden.

\section{Theorie}
	\label{sec:theorie}

	Die Reibungskraft $\vec{F}$ eines sich bewegenden K"orpers in einer Fl"ussigkeit ist von der Ber"uhrungsfl"ache $A$ und der Geschwindigkeit $v$ abh"angig.
	Die Viskosit"at $\eta$ wird mit einem Kugelfallviskosimeter bestimmt, in welchem eine Kugel in destilliertem Wasser f"allt, ohne Wirbel auszubilden.
	Da es sich um eine laminare Str"omung handelt gilt die Stokes'sche Reibung:

	\begin{equation}
		F_R = 6\, \pi\, \eta\, v\, r.
	\end{equation}

	Mit zunehmender Fallgeschwindigkeit $v$ nimmt die Reibung zu, bis sich ein Kr"af\-te\-gleich\-ge\-wicht zwischen Reibungskraft $\vec{F}_\mathrm{R}$, Schwerkraft $\vec{F}_\mathrm{g}$ und Auftrieb $\vec{F}_\mathrm{A}$ einstellt.
	Danach f"allt die Kugel mit konstanter Geschwindigkeit.
	Die Viskosit"at $\eta$ l"a"st sich nun berechnen durch:

	\begin{equation}
		\eta = K\,(\rho_\mathrm{K} - \rho_{\mathrm{Fl}})\,t. \label{viskositaet_0}
	\end{equation}

	\begin{center}
			\tiny{K = Proportionalit"atskonstante, $\rho_\mathrm{K}$ = Dichte der Kugel, $\rho_{\mathrm{Fl}}$ = Dichte der Fl"ussigkeit, t = Fallzeit}
	\end{center}

	Die Temperaturabh"angigkeit der Viskosit"at wird durch die Andradesche Gleichung

	\begin{equation}
		\eta(T) = A \exp(\frac{B}{T}), \label{viskositaet_dyn}
	\end{equation}

	beschrieben, wobei $A$ und $B$ Konstanten sind.