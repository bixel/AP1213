\section{Diskussion}
\label{sec:diskussion}

Die Ergebnisse des Versuchs stimmen teilweise gut mit den Literaturwerten "uberein.
So ergab sich f"ur die Ionisationsspannung $U_\mathrm{ion} \approx \SI{11.75}{\volt}$ anstatt $\SI{10.44}{\electronvolt}$. F"ur den Anregungsstrom aus Aufgabe a) des Quecksilbers ergab sich $U_\mathrm{a} = \SI{5}{\volt}$ statt $U_\mathrm{a} = \SI{4.9}{\volt}$.

Jedoch gab es bei der Durchf"uhrung leichte Schwierigkeiten bei der Handhabung des XY-Schreibers. Nachdem das Prinzip verstanden wurde wurden jedoch gute Ergebnisse erzielt.

Bei der Auswertung war es jedoch relativ ungenau die Werte aus den Graphiken abzulesen. 


\begin{thebibliography}{9}
	\bibitem{anleitung} Physikalisches Anf"angerpraktikum der TU Dortmund: Versuch V601 - Der Franck-Hertz-Versuch. \url{http://129.217.224.2/HOMEPAGE/PHYSIKER/BACHELOR/AP/SKRIPT/V601.pdf}. Stand: Juni 2013.

	\bibitem{anregung} Anregungsenergie von Quecksilber. \url{http://lp.uni-goettingen.de/get/text/1612}.

	\bibitem{ion} Ionisationsenergie von Quecksilber. \url{http://www.uni.merkertweb.de/p2/franck-hertz_v.pdf}.
\end{thebibliography}
