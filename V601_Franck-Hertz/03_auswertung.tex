\section{Auswertung}
\label{sec:auswertung}

\subsection{Mittlere Wegl"ange} % (fold)
\label{sub:mittlere_wegl_ange}


\begin{table}[!h]
\begin{center}
\begin{tabular}{|r|r|r|r|}
\hline
T[$\SI{}{^\circ}$] & $p_\mathrm{s"at}[\SI{}{\milli\bar}]$ & $\bar{w}[\SI{}{\centi\meter}]$ & $\frac{a}{\bar{w}}$\\
\hline
\hline
25  & 0.0052  & 0.5532 & 1.81\\
105 & 0.6924  & 0.0042 & 238.75\\
150 & 4.7948  & 0.0006 & 1653.38\\
190 & 19.5284 & 0.0001 & 6733.94\\
\hline
\end{tabular}
\caption[]{Errechnete Werte f"ur $p_\mathrm{s"at}$ und $\bar{w}$ in Abh"angigkeit von der Temperatur.}
\label{tab:weg}
\end{center}
\end{table}

Das Verh"altnis von der L"ange der verwendeten R"ohre $a$ zu der freien Wegl"ange der Elektronen $\bar{w}$ soll etwa einen Faktor von $1000 - 4000$ ergeben.
Nach den Formeln \eqref{p} und\eqref{w} ergaben sich die Werte aus Tabelle \ref{tab:weg} f"ur $p_\mathrm{s"at}$ und $\bar{w}$ mit $a = \SI{1}{\centi\meter}$.

Es ist zu erkennen, dass die Temperaturen $\SI{25}{^\circ}$ und $\SI{105}{^\circ}$ nicht f"ur eine Franck-Hertz Kurve geeignet sind, da das Verh"altnis zu niedrig ist.
Bei der Temperatur von $\SI{190}{^\circ}$ ist es schon etwas zu hoch, jedoch lassen sich noch akzeptable Ergebnisse erzielen, w"ahrend bei $\SI{150}{^\circ}$ das Verh"altnis sehr gut ist.

\subsection{Energieverteilung der beschleunigten Elektronen} % (fold)
\label{sub:energieverteilung_der_beschleunigten_elektronen}

\subsubsection{$\si{25}{^\circ C}$} % (fold)
\label{sub:_si}


\begin{table}[!h]
\begin{center}
\begin{tabular}{|r|r|}
\hline
$U_\mathrm{a}$[V] & $\Delta y$ \\
\hline
\hline

7.000 	& 0.1 \\
7.125 	& 0.1 \\
7.250 	& 0.1 \\
7.375 	& 0.1 \\
7.500 	& 0.1 \\
7.625 	& 0.1 \\
7.750 	& 0.1 \\
7.875 	& 0.1 \\
8.000 	& 0.1 \\
8.125 	& 0.1 \\
8.250 	& 0.2 \\
8.375 	& 0.2 \\
8.500 	& 0.3 \\
8.625 	& 0.3 \\
8.750 	& 0.4 \\
8.875 	& 0.3 \\
9.000 	& 0.4 \\
9.125   & 0.4 \\
9.250 	& 0.6 \\
9.375 	& 0.8 \\
9.500 	& 1.2 \\
9.625 	& 2.0 \\
9.750 	& 3.8 \\
9.875 	& 1.5 \\

\hline
\end{tabular}
\caption[]{Wertepaare der Energieverteilung bei $\SI{25}{^\circ}$.}
\label{tab:a1}
\end{center}
\end{table}

\begin{figure}[!h]
	\centering
	\includegraphics[width = 10cm]{img/t20.pdf}
	\caption{Energieverteilung der beschleunigten Elektronen bei $\SI{25}{^\circ}$.}
	\label{gra}
\end{figure}

Bei der Messung der Energieverteilung der beschleunigten Elektronen bei Zimmertemperatur ergab sich ein maximaler Anstieg der Kurve bei:

\begin{equation}
	U_\mathrm{a,max} = \SI{9.75}{\volt}
\end{equation}

Mithilfe der Formel:

\begin{equation}
	U_\mathrm{b} - U_\mathrm{a,max} = K
\end{equation}

ergibt sich f"ur $U_\mathrm{b} = \SI{11}{\volt}$:

\begin{equation}
	K = \SI{1.25}{\volt}
\end{equation}

Qualitativ l"asst sich sagen, dass der gr"o"ste Teil der Elektronen Energien von $\SI{9.25}{\volt} - \SI{10}{\volt}$ besitzt.

\subsubsection{$\si{150}{^\circ C}$} % (fold)
\label{sub:_si}


\begin{table}[!h]
\begin{center}
\begin{tabular}{|r|r|}
\hline
$U_\mathrm{a}$[V] & $\Delta y$ \\
\hline
\hline

1.00 &	0.8  \\
1.25 &	0.8  \\
1.50 &	0.8  \\
1.75 &	0.8  \\
2.00 &	0.9  \\
2.25 &	0.8  \\
2.50 &	1.0  \\
2.75 &	0.8  \\
3.00 &	0.8  \\
3.25 &	0.9  \\
3.50 &	0.8  \\
3.75 &	0.8  \\
4.00 &	0.5  \\
4.25 &	0.4  \\
4.50 &	0.4  \\
4.75 &	0.1  \\
5.00 &	0.0  \\
5.25 &	0.0  \\
5.50 &	0.0  \\
5.75 &	0.0  \\

\hline
\end{tabular}
\caption[]{Wertepaare der Energieverteilung bei $\SI{150}{^\circ}$.}
\label{tab:a1}
\end{center}
\end{table}

\begin{figure}[!h]
	\centering
	\includegraphics[width = 10cm]{img/t150.pdf}
	\caption{Energieverteilung der beschleunigten Elektronen bei $\SI{150}{^\circ}$.}
	\label{gra}
\end{figure}
