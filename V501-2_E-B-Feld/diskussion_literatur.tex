\newpage
\section{Diskussion}
	\label{sec:diskussion}

	Allgemein l"asst sich sagen, dass die Werte teilweise sehr Nahe an die Literaturwerte herankommen. Daher hat der Versuch sein Ziel erreicht und einem klar gemacht, wie sich E- und B-Felder auf Elektronen auswirken. 

	Einen Einfluss auf alle Daten d"urften die vielen elektrischen Ger"ate und Leitungen im Geb"aude gehabt haben.

	Bei Versuch 501a ergab sich aus den Messwerten $a = \SI{35.848 (947)}{\centi\meter}$. F"ur die Formel des Proportionalit"atsfaktors ergab sich jedoch $\SI{46.84}{\centi\meter}$. M"ogliche Fehlerquelle k"onnte die anteilige Mittelung des nicht konstanten Plattenabstandes $d$ sein, aber auch die Messungenauigkeit bei der Ablenkung D. Diese war nicht immer ganz eindeutig, da sie je nach Ablenkspannung andere Formen annahm. Weiterhin k"onnte die R"ohre nicht mehr den angegebenen Restdruck haben und so das Ergebnis verf"alschen.

	Bei Versuch 501b ergab sich der Wert $\SI{79.467 (98)}{\hertz}$. Laut Anzeige auf dem Si\-nus\-span\-nungs\-ge\-ne\-ra\-tors sollte dieser eine Frequenz von $80 - 90 \SI{}{\hertz}$ haben. Da es wirklich nur einen sehr kleinen Bereich gab, indem die Sinusspannung auf der Anzeige wirklich stehend war, k"onnten sich hier leichte Ablesefehler ergeben haben, wodurch sehr nah an die 80 $\SI{}{\hertz}$ herangekommen wird.

	F"ur die spezifische Ladung der Elektronen ergab sich der gemittelte Wert von $\SI{1.9225}{\coulomb\per\kilo\gram}*10^{11}$ welcher nicht weit von dem Literaturwert $\SI{1.758}{\coulomb\per\kilo\gram}*10^{11}$ abweicht. M"ogliche Mess\-un\-ge\-nau\-ig\-keit\-en sind an dem Amperemeter aber auch am Voltmeter der B\-schleu\-ni\-gungs\-span\-nung nicht auszuschlie"sen. Auch k"onnte die relativ kleine Helmholtzspule ein nicht vollkommen homogenes Feld erzeugt haben, wodurch das Ergebnis verf"alscht worden w"are.

	In Aufgabenteil 502b ergab sich f"ur die Horizontalkomponente $B_\mathrm{hor} = \SI{16.58}{\micro\tesla}$ und den Inklanationswinkel $\varphi = 70$ Grad. Daraus folgte f"ur die Totalintensit"at $B_\mathrm{total} = \SI{48.48}{\micro\tesla}$. Die Literaturwerte f"ur Mitteleuropa liegen bei $B_\mathrm{hor} = \SI{20}{\micro\tesla}$, $\varphi = 63 - 70$ Grad und $B_\mathrm{total} = \SI{48}{\micro\tesla}$. Dabei nimmt die Intensit"at zu, je weiter man sich auf die magnetischen Pole zubewegt. 
	Somit liegen unsere Werte sehr nah an den Literaturwerten. Es sind "au"sere magnetische Einfl"usse durch die eletrischen Leitungen im Haus jedoch keinenfalls auszuschlie"sen, da wir schon beim herumprobieren mit dem Deklinatorium-Inklinatorium an verschiedenen Stellen des Raumes auch verschiedene Richtungen f"ur Norden fanden. Zudem ist dieses nicht mehr das neueste Ger"at und musste noch extra fixiert werden, damit die Nadel nicht komplett eingeklemmt wird.


\section{Literatur}

	Alle Grafiken wurden eigenst"andig mit Gnuplot oder pyplot erstellt oder aus der Ver\-suchs\-an\-lei\-tung "`Ablenkung eines Elektronenstrahls im elektrischen Feld"' und "`Ablenkung eines Elektronenstrahls im transversalen Magnetfeld"' der TU Dortmund (Stand 29.10.12) entnommen.