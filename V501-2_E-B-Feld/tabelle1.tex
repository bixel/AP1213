\begin{table}[h]
\begin{center}
\begin{tabular}{c|c|c||c|c|c||c|c|c}
Ub[V] & Ud[V] & D[1/4 in] & Ub[V] & Ud[V] & D[1/4 in] & Ub[V] & Ud[V] & D[1/4 in] \\
\hline
200 & -7,1 & 4 & 250 & -9,3 & 4 & 300 & -11,9 & 4 \\
200 & -3,9 & 3 & 250 & -5,2 & 3 & 300 & -7,1 & 3 \\
200 & -0,1 & 2 & 250 & -0,4 & 2 & 300 & -1,2 & 2 \\
200 & 3,4 & 1 & 250 & 4 & 1 & 300 & 4,3 & 1 \\
200 & 6,9 & 0 & 250 & 8,4 & 0 & 300 & 9,5 & 0 \\
200 & 10,5 & -1 & 250 & 12,6 & -1 & 300 & 14,8 & -1 \\
200 & 13,7 & -2 & 250 & 17,3 & -2 & 300 & 20,4 & -2 \\
200 & 17,4 & -3 & 250 & 21,4 & -3 & 300 & 25,3 & -3 \\
200 & 20,7 & -4 & 250 & 25,4 & -4 & 300 & 30,0 & -4 \\
\end{tabular}
\caption{Messwerte zu Aufgabe a bei verschiedenen Beschleunigungsspannungen}
\label{tabelle_1}
\end{center}
\end{table}