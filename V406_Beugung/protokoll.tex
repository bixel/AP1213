%Dokumentklasse Festlegen
\documentclass[paper = a4, ngerman]{scrartcl}

	%ngerman fur Umlaute, Anf"hrungszeichen, etc.%
	\usepackage[ngerman]{babel, varioref}

	%siunitx fur SI-Einheitesnsystem%
	\usepackage{siunitx}
	\sisetup{
	    locale=DE,
	    separate-uncertainty=true,
    	per-mode=fraction
	}

	%graphicx fur Bildeinbindung%
	\usepackage{graphicx}

	%enthaelt eqnref{}-Operator
	\usepackage{amsmath}

	%Links innerhalb und ausserhalb des Dokuments
	\usepackage{hyperref}

	%Formatiert urls
	\usepackage{url}

	%Bibliographie
	\usepackage[numbers]{natbib}

		%Bibtex: Nachnamen in Kapitälchen
		%\renewcommand*{\mkbibnamelast}[1]{\textsc{#1}}
		\newcommand*{\mkbibnamelast}[1]{\textsc{#1}}

		% Makros für Anhang + Referenzen
		\newcommand{\anhang}{
			\clearpage		% Anhang auf eine extra Seite packen
			\setcounter{page}{0}	
			\pagenumbering{Roman}	% Anhang wird in römischen Seitenzahlen numeriert
			\appendix		% Kapitelnummerierung in Großbuchstaben statt Zahlen.
		}

		\newcommand{\referenzen}{
			\bibliographystyle{alphadin} 			% Alphabetisch sortiert im DIN-Format
			\addcontentsline{toc}{section}{Referenzen}
			\phantomsection					% Referenzen ins Inhaltsverzeichnis
			\renewcommand{\refname}{\section*{Referenzen}\vspace*{-1em}} % Benennt das Kapitel um
			\bibliography{../bib.bib} 	% Die BibTeX-Datei einbinden
		}

	% Seitenlayout ändern mit Fancy
	\usepackage{fancyhdr}	% Paket zum bequemeren Verändern des Seitenlayouts

		% Tabellen ändern:
			\renewcommand{\thetable}{\arabic{section}.\arabic{table}} % figures bekommen die richtige Nummerierung: x.y
			\makeatletter \@addtoreset{table}{section} \makeatother      % nach jeder section wird neu gezählt

		% Kapitelüberschriften in der Kopfzeile:
			\renewcommand*{\sectionmark}[1]{\markboth{}{\thesection\ #1}}
			%\renewcommand*{\subsectionmark}[1]{\markboth{}{\thesubsection\ #1}}
			\renewcommand*{\subsectionmark}[1]{\markboth{}{}} % keine Unterüberschriften in der Kopfzeile
			\renewcommand{\plainheadrulewidth}{0.4pt}
		
		% Seitennummern rechts in der Kopfzeile:
			\lhead[\fancyplain{\thepage}{\thepage}]{\fancyplain{}{\rightmark}}
			\rhead[\fancyplain{}{\leftmark}]{\fancyplain{\thepage}{\thepage}}
		
		%Fußzeilen bleiben leer
			\lfoot{}
			\cfoot{}
			\rfoot{}

	%Texteinzug vor Absatz entfernen%
	\parindent 0pt

	%Titelseiten Parameter
	\title{V406 - Beugung am Spalt}
	\date{27. Oktober 2012}
	\author{Kevin Heinicke und Markus Stabrin}

\begin{document}

	%\maketitle

	% ANFANG Titelseite %
	\vspace*{3cm}

	\begin{center}
		\large
		TU Dortmund
	\end{center}

	\begin{center}
		\Huge
		V406 - Beugung am Spalt
	\end{center}

	\vspace{6cm}
	\begin{center}
		\begin{minipage}[b]{8cm}
			\Large
			Markus Stabrin \\
			\normalsize
			markus.stabrin@tu-dortmund.de \\

			\Large
			Kevin Heinicke\\
			\normalsize
			kevin.heinicke@tu-dortmund.de \\
			\\
			\\

			Versuchsdatum: 23. Oktober 2012 \\
			\\
			Abgabedatum: 30. Oktober 2012
		\end{minipage}
	\end{center}

	% ENDE Titelseite %

	\newpage

	\section{Einleitung}
	\label{sec:einleitung}
	In diesem Versuch wird das Verhalten von monochromatischem Licht bei Beugung an d"unnen Spalten untersucht.\\
	Unter der Annahme, dass sich Licht wellenartig ausbreitet wird der Intensit"atsverlauf des Beugungsmusters hinter einem Einfach-, sowie einem Doppelspalt untersucht.
	Aus den Messwerten l"asst sich schlie"slich eine Aussage "uber die Spaltbreite machen.

\section{Theorie}
	\label{sec:theorie}

	Im Folgenden werden verschiedenen Annahmen gemacht, die die Beschreibung des Lichtes vereinfachen.
	Dadurch k"onnen die Ph"anomene dieses Experimentes gut erkl"art werden.

	\subsection{Das Huygenssche Prinzip}

		Um die Natur des Lichtes detailliert beschreiben zu k"onnen, muss man es quan\-ten\-me\-cha\-nisch betrachten.
		F"ur etliche Ph"anomene reicht es jedoch aus, "uber gro"se Zahlen von Lichtquanten zu mitteln und diese durch das klassische Wellenmodell n"aherungsweise zu beschreiben.
		Diese N"aherung wird hier gemacht.

		Das Huygenssche Prinzip geht von der Welleneigenschaft des Lichtes aus.
		Es besagt, dass jeder Punkt einer Wellenfl"ache zur gleichen Zeit Elementarwellen aussendet.
		Diese Kugelwellen interferieren miteinander und bilden eine neue Wellenfront,
		die die Ein\-h"ul\-len\-de der Elementarwellen ist und wiederum neue Elementarwellen aussendet.
		Die "Uberlagerung aller Elementarwellen an einem Ort im Raum beschreibt dann den dortigen Schwingungszustand der Welle.

	\subsection{Fresnel- und Fraunhoferbeugung}
		\label{subsec:beugung}

		F"ur die Beschreibung von Beugungserscheinungen gibt es grunds"atzlich zwei verschiedene Versuchsanordnungen. Abb. \ref{abb_fresnel_fraunhofer} skizziert diese.

		\subsubsection{Fresnelbeugung}
			\label{subsubsec:fresnel}

			Die Fresnelsche Anordnung betrachtet eine Lichtquelle, die sich im endlichen Abstand vor dem Spalt befindet.
			Dadruch divergieren die Strah\-len\-b"un\-del und das Licht wird am Spalt in unterschiedlichen Winkeln gebeugt.
			Schlu"sfolgerungen auf den Versuchsaufbau durch Messung des Intensit"atsverlaufes werden damit sehr schwierig.

		\subsubsection{Fraunhoferbeugung}
			\label{subsubsec:fraunhofer}

			Diese Anordnung geht von parallelen Lichtb"undeln aus, die von einer unendlich weit entferten Lichtquelle entsandt werden.
			Hierdurch werden Lichtb"undel gleicher Phase im gleichen Winkel abelenkt.
			Die Beschreibung wird hier wesentlich einfacher, weil lediglich der Gangunterschied bei \emph{einem} Winkel, betrachtet werden muss.
			Die unendlich entfernte Lichtquelle l"asst sich gut durch einen Laser realisieren.

			Aus diesem Grund und weil der Fraunhoferaufbau eine leichtere Beschreibung der Beugung liefert, wird der Versuch damit durchgef"urt.

			\begin{figure}[h]
				\centering
				\includegraphics[width = 14cm]{fresnel_fraunhofer.jpg}
				\caption{Unterschied zwischen Fresnelschem und Fraunhoferschem Versuchsaufbau}
				\label{abb_fresnel_fraunhofer}
			\end{figure}

	\subsection{Beugungsmuster am Einzelspalt}
		\label{subsec:herleitung}
		\label{subsec:muster_einzelspalt}

		\begin{figure}[h]
				\centering
				\includegraphics[width = 9cm]{spalt.jpg}
				\caption{Einzelspalt}
				\label{abb_spalt}
		\end{figure}

		Betrachtet man nun den Spalt genauer, l"asst sich die Phasendifferenz $\delta$ zweier Strah\-len\-b"un\-del einfach beschreiben. Die durch Elementarwellen bestimmten B"undel haben dann unter dem Winkel $\varphi$ die Phasendifferenz $\delta$

		\begin{equation}
			\delta = \frac{2 \pi s}{\lambda} = \frac{2 \pi x \sin(\varphi)}{\lambda} .
		\end{equation}

		Hierbei ist $s$ gerade der Gangunterschied der B"undel und $\lambda$ die Wellenl"ange des ein\-fal\-len\-den Lichtes.\\
		Um die Amplitude $B(z, t, \varphi)$ zu bestimmen muss "uber alle Elementarwellen -- also "uber die gesamte Spaltbreite $b$ -- integriert werden:

		\begin{eqnarray}
			B(z, t, \varphi) &=& A_0 \int \limits_0^b \exp{\left[i \left(\omega t - \frac{2 \pi z}{\lambda} + \delta \right)\right]} \mathrm{dx} \nonumber \\
			&=& A_0 \exp{\left[i \left(\omega t - \frac{2 \pi z}{\lambda}\right)\right]} \int \limits_0^b \exp{\left(\frac{2 \pi i x \sin{\varphi}}{\lambda}\right)} \mathrm{dx} \nonumber\\
			\Rightarrow \enspace
			B(z, t, \varphi) &=& A_0 \exp{\left[ i \left( \omega t - \frac{2\pi z}{\lambda}
			\right) \right] } 
			\exp{\left[i \left( \frac{\pi b \sin{\varphi}}{\lambda} \right)\right]}  \cdot \nonumber \\
			&\cdot& \frac{\lambda}{\pi \sin{\varphi}}
			\sin{\left( \frac{\pi b \sin{\varphi}}{\lambda} \right)}
			.
			\label{eqn:allgemeine_lsg}
		\end{eqnarray}

		Dies beschreibt die Beugung an einem Parallelspalt.
		Die ersten beiden exponentiellen Anteile aus Gleichung \eqref{eqn:allgemeine_lsg} beschreiben die Amplitude in Abh"angigkeit der Zeit $t$ und des Ortes $z$ (senkrecht auf der Schirmfl"ache).

		Die einfallenden Lichtb"undel werden "uber lange Zeit gemessen.
		Wegen der hohen Frequenz $\omega$ des Lichtes k"onnn wir nur die "uber einige Zeit gemittelte Intensit"at $I$ messen. Der Anteil der Zeitabh"angigkeit f"allt also weg.

		Zudem bewegen wir uns lediglich entlag der Schirmfl"ache -- also in x-Richtung --, wodurch auch der zweite, z-abh"angige Term nicht betrachtet werden muss.
		Dann gilt:

		\begin{eqnarray}
			I(\varphi) & \propto & B^2(\varphi) 
		 . \nonumber
		\end{eqnarray}

		Mit einer Abk"urzung l"asst sich der Zusammenhang "ubersichtlich darstellen:

		\begin{eqnarray}
			\eta(\varphi) & := & \frac{\pi b \sin{\varphi}}{\lambda} , \nonumber \\
			\Rightarrow \qquad
			B(\varphi) & = & A_0 b \frac{\sin{\eta}}{\eta} , \label{amplitude_einfach} \\
			\Rightarrow \qquad
			I(\varphi) & \propto & A_0^2 b^2 \left(\frac{\sin{\eta}}{\eta}\right)^2  .
			\label{prop_einzelspalt}
		\end{eqnarray}

		Man erkennt, dass die H"ohe der Maxima n"aherungsweise mit dem Quadrat des Winkels abnimmt.
		Ausserdem entsteht ein Hauptmaximum bei $\varphi = 0$. Symmetrisch auf beiden Seiten entstehen etliche Nebenmaxima.
		Ein Minimum liegt vor, wenn $I(\varphi) = 0$ ist. Das gilt f"ur

		\begin{equation}
			\sin{\varphi_n} = \pm \, n \frac{\lambda}{b} \quad , \quad n = \left(1, 2, \dots\right) .
		\end{equation}

		\begin{figure}[h]
			\centering
			\includegraphics[height = 10cm]{theorie_1.png}
			\caption{Theoretischer Intensit"ats- und Amplitudenverlauf}
		\end{figure}

	\subsection{Fourier-Transformation}
		\label{subsec:fourier}

		Die Amplitudenverteilung l"asst sich bei diesem Aufbau auch durch eine 
		Four\-ier-Trans\-for\-ma\-tion der einzelnen Amplitudenverteilungen der einfallenden Wellen beschreiben.
		Transformiert man die Funktion $f(x)$ mit

		\begin{equation}
			f(x) = \left\{
			\begin{array}{ll}
			 	 A_0 & \qquad 0 \leq x \leq b \\
			 	 0 & \qquad \mathrm{sonst} 
			 \end{array} ,
			 \right.
		\end{equation}

		erh"alt man f"ur die Fourier-Transformation

		\begin{equation}
			g(\xi) := \int \limits_{-\infty}^{\infty} f(x) e^{ix \xi} \mathrm{dx} =
			\frac{A_0}{i \xi} \left(-1 + e^{i \xi b}\right) .
			\label{amplitude_fourier}
		\end{equation}

		Unter Anwendung der Eulerschen Formel und mit

		\begin{equation}
			g(\xi) = \frac{2 \pi \sin{\varphi}}{\lambda}
		\end{equation}

		stimmen \ref{amplitude_fourier} und \ref{amplitude_einfach} "uberein.
		Das Huygensche Prinzip wird hierdurch also mathematisch formuliert.
		Das Integral beschreibt dabei die Summation "uber alle Elementarwellen.
		Die Aperturfunktion $f(x)$ kann zudem hierbei einfach um eine Variable $y$ erweitert werden,
		was die Beschreibung der Beugung an zweidimensionalen Objekten erm"oglicht. \\
		F"ur die Auswertung dieses Experimentes ist besonders wichtig, dass sich die Fourier-Transformation umkehren l"asst.
		Wir k"onnen somit aus den Messwerten der Intensit"at $I(\varphi)$,
		die proportional zu $B^2(\varphi)$ ist, auf die Gestalt der Aperturfunktion $f(x)$
		--  also auf unseren Spalt -- schlie"sen.
		
	\subsection{Beugungsmuster am Doppelspalt}
		\label{sec:muster_doppelspalt}

		Der Doppelspalt l"asst sich als "Uberlagerung zweier Einzelspalte im Abstand $s$ beschreiben.
		Hieraus bekommt man zus"atzlich zur Abh"angigkeit \eqref{prop_einzelspalt} einen Cosinus-Anteil,
		der den Gangunterschied durch den Spaltabstand ber"ucksichtigt:

		\begin{equation}
			I(\varphi) \propto B^2(\varphi) = 
			4 \cos^2{\left( \frac{\pi s \sin{\varphi}}{\lambda} \right)}
			\left( \frac{\lambda}{\pi b \sin{\varphi}} \right)^2
			\sin^2{\left( \frac{\pi b \sin{\varphi}}{\lambda} \right)} .
			\label{prop_doppelspalt}
		\end{equation}

		\begin{figure}[h]
			\centering
			\includegraphics[width = 15cm]{theorie_2.png}
			\caption{Theoretischer Amplitudenverlauf des Doppelspaltes}
			\label{theoriekurve_doppelspalt}
		\end{figure}


	\newpage

	\section{Aufbau und Durchf"uhrung}
	\label{sec:durchfueuhrung}
	F"ur den Versuch stand ein He-Ne-Laser, ein gr"o"senverstellbarer Einfachspalt , ein gr"o\-"sen\-ver\-stell\-bar\-er Doppelspalt, ein lichtempfindlicher Detektor auf einer mechanischen Schiene sowie ein Amperemeter zur verf"ugung.

	\subsection{Messaufgaben}
		\begin{enumerate}
			\item \label{aufg_1} Punktweises Ausmessen der Beugungsfigur des Einfachspalts (50 Messpunkte). Anschlie"sendes Bestimmen der Spaltbreite $b$ und Ausmessen der Spaltbreite $b$ mit dem Mikroskop.

			\item \label{aufg_2} Wie \ref{aufg_1} aber mit variablem Einfachspalt und ohne Mikroskop. 

			\item \label{aufg_3} Wie \ref{aufg_1} aber mit festem Doppelspalt und anschlie"sendem Vergleich mit der theo\-re\-tisch\-en Verteilung des Einfachspaltes.
		\end{enumerate}

	\subsection{Durchf"uhrung}
		\label{sec:durchfuehrung}
		\subsubsection{Messungen}
			\label{sec:messung}

			Der Laser beleuchtet mit einer Wellenl"ange von $\lambda = \SI{633}{\nano \meter}$ wie in \ref{Versuchsaufbau} beschrieben verschiedene Einzel- und Doppelspalten mit Spaltbreiten von ca. 20 -- 200$\SI{}{\micro \meter}$. Ein lichtempfindlicher Detektor befindet sich in $\SI{100}{\centi \meter}$ entfernung vom Spalt, welcher senkrecht zur optischen Achse verstellbar ist.

			\begin{figure}[h]
					\centering
					\includegraphics[width = 14cm]{Versuchsaufbau.png}
					\caption{Versuchsaufbau}
					\label{Versuchsaufbau}
			\end{figure}

			Anfangs muss die Apparatur justiert werden. Daf"ur wird der Laser fixiert und die h"ohe des Detektors in Mittelstellung so ausgerichtet, dass der Laserstrahl genau auf den Sensor trifft. Anschlie"send wird der f"ur die Messung ben"otigte Spalt in die daf"ur vorgesehene Vorrichtung gesteckt und so daran gedreht bis das Hauptmaximum mittig vom Sensor erfasst wird und die n-ten Nebenmaxima links und rechts des Hauptmaximas in etwa die gleiche Intensit"at haben.

			Nun kann die Intensit"at der Beugungsfigur abh"angig von der Detektorstellung gemessen werden. Dabei wird "uber einen Verschiebeweg von $\SI{50}{\milli\meter}$ die Intensit"at punktweise gemessen. Die genaue Position kann an der Skala der Schiene abgelesen werden. Ein Strich entspricht einem $\SI{}{\centi \meter}$, genauso wie eine Trommelumdrehung. Auf der Trommel kommt eine Skala hinzu, welche in $\SI{}{\micro \meter}$ ablesbar ist.

			Aus der Position des Detektors l"asst sich der Beugungswinkel $\varphi$ aus der Detektorstellung $\zeta$ bestimmen. Dies ist notwendig um die aufgenommene Intensit"atskurve $I(\zeta)$ mit dem Verlauf $I(\varphi)$ vergleichen zu k"onnen.

			Es gilt:

			\begin{equation}
				\varphi \approx \tan{\varphi} = \frac{\zeta - \zeta_0}{L}
			\end{equation}

			Dabei ist $\zeta_0$ die Detektorstellung f"ur den ungebeugten Strahl und $L$ der Abstand des Spalts zur Detektoblende.

			F"ur den Doppelspalt wird genauso vorgegangen.

			Zu beachten ist, dass auch bei abgeschaltetem Laser bereits ein sogenannter Dunkelstrom $I_{\text{du}}$ flie"st, welcher daher bei abgedeckter Detektorblende gemessen werden muss.

		\subsubsection{Messung des Abstands per Mikroskop}
			\label{sec:messung_mikro}

			Zun"achst muss das Mikroskop geeicht werden, damit die Breite $b$ des Spalts ausgemessen werden kann. Dies geschieht mithilfe eines Objektmikrometers. Dies ist eine Glasplatte mit einge"atzter Mikrometerskala. In der Brennebene des Okulars liegt eine Skala mit willk"urlicher Teilung. 

			Nach der Fokussierung auf die Skala das Objektmikrometer k"onnen die Skalen verglichen werden und so die Teilung in $\SI{}{\micro \meter}$ ausgedr"uckt werden.


	\newpage

	\section{Auswertung}
	\label{sec:Auswertung}

	\subsection{Bestimmung des Dunkelstroms}
		\label{sub:bestimmung_des_dunkelstroms}
		
		Um die Messwerte mit der Theorie vergleichen zu k"onnen muss zuerst der Dunkelstrom $I_\mathrm{du}$ gemessen werden.

		Dazu wurde unter Versuchsbedingungen, jedoch mit ausgeschaltetem Laser, eine Messung durchgef"uhrt.

		F"ur den Dunkelstrom ergab sich:

		\begin{equation}
			I_\mathrm{du} = \SI{0.13}{\nano \ampere}
		\end{equation}

		Die Aufgetragenenen Messwerte in den Tabellen sind noch nicht um diesen Wert reduziert.

	\subsection{Bestimmung der Spaltbreite eines festen Einfachspalts}
		\label{sub:bestimmung_der_spaltbreite_eines_festen_einfachspalts}
		
		Im folgenden wird die Spaltbreite eines Einfachspalts auf zwei verschiedene Weisen gemessen. Einmal mithilfe des Beugungsmusters und einmal mit Hilfe eines Mikroskops.

		\subsubsection{Bestimmung mit Hilfe des Beugungsmusters}
			\label{sub:Bestimmung_mit_Hilfe_des_Beugungsmusters}

			Wie die Spaltbreite bestimmt wird, wird bereits in \ref{sec:messung} erl"autert. Die bereinigten Messwerte finden sich in Tabelle \ref{tabelle_1} wieder.

			Die gemessene Intensit"at wurde in Grafik \ref{graf_1} gegen die Detektorposition aufgetragen und eine Theoriekurve nach Gleichung \ref{prop_einzelspalt} eingef"ugt um die Kurven vergleichen zu k"onnen.

			Zur Anpassung wurde PyPlot benutzt.

			Es ergibt sich mithilfe der Formel:

			\begin{equation}
				f(x) = a^2 b^2 \left(\frac{\lambda}{\pi b sin(\frac{x-x_\mathrm{0}}{d})}\right)^2 sin^2 \left( \frac{\pi b sin(\frac{x-x_\mathrm{0}}{d})}{\lambda} \right)
			\end{equation}

			f"ur die Werte ergaben sich:

			\begin{table}[h]	
\caption{Intensit"at des festen Einfachspalts abgh"angig von der Detektorstellung x}
\centering
\begin{tabular}{|l l|l l|l l|} \hline \hline
	x[mm] & I[nA] & x[mm] & I[nA] & x[mm] & I[nA]\\
	\hline
	0	&	1.3   &  17	&	12  & 34	&	2.75\\
	1	&	1.3   &  18	&	12.5& 35	&	3.4\\
	2	&	1.5   &  19	&	16  & 36	&	4\\
	3	&	1.8   &  20	&	24  & 37	&	3.6\\
	4	&	2.2   &  21	&	32  & 38	&	2.7 \\
	5	&	2.3   &  22	&	46  & 39	&	1.55\\
	6	&	2.1   &  23	&	58  & 40	&	0.84\\
	7	&	1.5   &  24	&	68  & 41	&	0.58\\
	8	&	0.9    &  25	&	72  & 42	&	0.82\\
	9	&	0.8    &  26	&	68  & 43	&	1.1\\
	10	&	1.6   &  27	&	58  & 44	&	1.45\\
	11	&	3.4   &  28	&	44  & 45	&	1.5\\
	12	&	6.4   &  29	&	28  & 46	&	1.26\\
	13	&	9.6   &  30	&	18  & 47	&	0.9\\
	14	&	11    &  31	&	9.2 & 48	&	0.66\\
	15	&	12.5  &  32	&	3.8 & 49	&	0.42\\
	16	&	12.5  &  33	&	2.4 & 50	&	0.34\\
	\hline \hline
\end{tabular}
\label{tabelle_1}
\end{table}


			\input


			blablabla auswertung

		\subsubsection{Bestimmung mit Hilfe des Mikroskops}
			\label{sub:Bestimmung_mit_Hilfe_des_mikroskops}

			Das Ausmessen mithilfe des Mikroskops wurde bereits in \ref{sec:messung_mikro} erkl"art.
			Es wurde auf das Objektmikrometer fokussiert. Nun wurde der Spalt unter das Objektiv gelegt und mithilfe eines verschiebbaren Teilstrichs die Spaltkanten "uberdeckt. Diese Fixierung wurde dann auf die Mikrometerskala des Objektmikrometers gelegt und ausgemessen.
			Aufgrund von Unsch"arfen wurde ein Fehler von $\sigma_0 = \SI{10}{\micro \meter}$ angenommen.

			Es ergab sich f"ur die Spaltbreite b:

			\begin{equation}
				b = \SI{70}{\micro \meter}
			\end{equation}

			Damit ergibt sich f"ur die gemessene Spaltbreite $b_\mathrm{mikro} = \SI{70 (10)}{\micro \meter}$.
			Es besteht ein gro"ser Unterschied zwischen der gemessenen und der angegebenen Spaltbreite von $b = \SI{80}{\micro \meter}$.
			Auf Fehlerquellen wird in der Diskussion eingegangen.

			\newpage

	\subsection{Bestimmung der Spaltbreite eines variablen Einfachspalts} 
		\label{sub:bestimmung_der_spaltbreite_eines_variablen_einfachspalts}
		
		Nun sollte die Breite $b$ eines variablen Einfachspalts gemessen werden. Das Messverfahren ist dasselbe wie zuvor.

		Die dazugeh"origen Intensit"atswerte abh"angig von der Detektorposition finden sich in Tabelle \ref{tabelle_2}. Weiter wurde eine Theoriekurve in die Grafik \ref{graph1} einge"ugt, sodass ein Vergleich zwischen Theorie und Experiment m"oglich ist. Daf"ur wurde Gnuplot verwendet.

		Es ergab sich:

		\begin{equation}
			b = \SI{7.46 (170)}{\micro \meter}
		\end{equation}

		\begin{table}[h]
\begin{center}
\begin{tabular}{c|c|c||c|c|c}
Ub[V] & Ud[V] & D[1/4 in] & Ub[V] & Ud[V] & D[1/4 in] \\
\hline
350 & -12,2 & 4 & 400 & -15,1 & 4 \\
350 & -5,9 & 3 & 400 & -8,2 & 3 \\
350 & 0 & 2 & 400 & -0,8 & 2 \\
350 & 6,4 & 1 & 400 & 5,8 & 1 \\
350 & 12,2 & 0 & 400 & 13,2 & 0 \\
350 & 18,1 & -1 & 400 & 20,2 & -1 \\
350 & 23,8 & -2 & 400 & 26,6 & -2 \\
350 & 29,7 & -3 & 400 & 33,7 & -3 \\
350 & 35,6 & -4 & 400 & 36 & -3,3 \\
\end{tabular}
\caption{Messwerte zu Aufgabe a bei verschiedenen Beschleunigungsspannungen}
\label{tabelle_2}
\end{center}
\end{table}


		\begin{figure}[h]
			\centering
			\includegraphics[width = 14cm]{graph1.jpg}
			\caption{Graphische Darstellung der des festen Einfachspalt}
			\label{graph1}
		\end{figure}

	\newpage

	\newpage
\section{Diskussion}
	\label{sec:diskussion}

	Allgemein l"asst sich sagen, dass die Werte teilweise sehr Nahe an die Literaturwerte herankommen. Daher hat der Versuch sein Ziel erreicht und einem klar gemacht, wie sich E- und B-Felder auf Elektronen auswirken. 

	Einen Einfluss auf alle Daten d"urften die vielen elektrischen Ger"ate und Leitungen im Geb"aude gehabt haben.

	Bei Versuch 501a ergab sich aus den Messwerten $a = \SI{35.848 (947)}{\centi\meter}$. F"ur die Formel des Proportionalit"atsfaktors ergab sich jedoch $\SI{46.84}{\centi\meter}$. M"ogliche Fehlerquelle k"onnte die anteilige Mittelung des nicht konstanten Plattenabstandes $d$ sein, aber auch die Messungenauigkeit bei der Ablenkung D. Diese war nicht immer ganz eindeutig, da sie je nach Ablenkspannung andere Formen annahm. Weiterhin k"onnte die R"ohre nicht mehr den angegebenen Restdruck haben und so das Ergebnis verf"alschen.

	Bei Versuch 501b ergab sich der Wert $\SI{79.467 (98)}{\hertz}$. Laut Anzeige auf dem Si\-nus\-span\-nungs\-ge\-ne\-ra\-tors sollte dieser eine Frequenz von $80 - 90 \SI{}{\hertz}$ haben. Da es wirklich nur einen sehr kleinen Bereich gab, indem die Sinusspannung auf der Anzeige wirklich stehend war, k"onnten sich hier leichte Ablesefehler ergeben haben, wodurch sehr nah an die 80 $\SI{}{\hertz}$ herangekommen wird.

	F"ur die spezifische Ladung der Elektronen ergab sich der gemittelte Wert von $\SI{1.9225}{\coulomb\per\kilo\gram}*10^{11}$ welcher nicht weit von dem Literaturwert $\SI{1.758}{\coulomb\per\kilo\gram}*10^{11}$ abweicht. M"ogliche Mess\-un\-ge\-nau\-ig\-keit\-en sind an dem Amperemeter aber auch am Voltmeter der B\-schleu\-ni\-gungs\-span\-nung nicht auszuschlie"sen. Auch k"onnte die relativ kleine Helmholtzspule ein nicht vollkommen homogenes Feld erzeugt haben, wodurch das Ergebnis verf"alscht worden w"are.

	In Aufgabenteil 502b ergab sich f"ur die Horizontalkomponente $B_\mathrm{hor} = \SI{16.58}{\micro\tesla}$ und den Inklanationswinkel $\varphi = 70$ Grad. Daraus folgte f"ur die Totalintensit"at $B_\mathrm{total} = \SI{48.48}{\micro\tesla}$. Die Literaturwerte f"ur Mitteleuropa liegen bei $B_\mathrm{hor} = \SI{20}{\micro\tesla}$, $\varphi = 63 - 70$ Grad und $B_\mathrm{total} = \SI{48}{\micro\tesla}$. Dabei nimmt die Intensit"at zu, je weiter man sich auf die magnetischen Pole zubewegt. 
	Somit liegen unsere Werte sehr nah an den Literaturwerten. Es sind "au"sere magnetische Einfl"usse durch die eletrischen Leitungen im Haus jedoch keinenfalls auszuschlie"sen, da wir schon beim herumprobieren mit dem Deklinatorium-Inklinatorium an verschiedenen Stellen des Raumes auch verschiedene Richtungen f"ur Norden fanden. Zudem ist dieses nicht mehr das neueste Ger"at und musste noch extra fixiert werden, damit die Nadel nicht komplett eingeklemmt wird.


\section{Literatur}

	Alle Grafiken wurden eigenst"andig mit Gnuplot oder pyplot erstellt oder aus der Ver\-suchs\-an\-lei\-tung "`Ablenkung eines Elektronenstrahls im elektrischen Feld"' und "`Ablenkung eines Elektronenstrahls im transversalen Magnetfeld"' der TU Dortmund (Stand 29.10.12) entnommen.

	\anhang

\end{document}