\section{Einleitung} % (fold)
\label{sec:einleitung}
	
	$\alpha$-Teilchen geben in Materie Energie durch elastische St"o"se, aber auch Anregungs- und Dissoziationsproze"se ab. In diesem Versuch soll nun bestimmt werden, welche Reichweite die $\alpha$-Strahlung in Luft bei verschiedenen Luftdr"ucken besitzt.
	
\section{Theorie} % (fold)
\label{sec:theorie}
	\subsection{Das Fouriersche Theorem und Fourieranalyse}
	\label{subsec:fourier}

\subsection{$\alpha$-Strahlung} % (fold)
\label{sub:_alpha_strahlung}

$\alpha$-Strahlung ist eine der drei Strahlungen, welche beim radioaktiven Zerfall instabiler Atomkerne auftritt.
Dabei sinkt die Kernladungszahl des Atoms und ein Heliumkern, die $\alpha$-Strahlung, wird emittiert.

\subsection{Reichweite von $\alpha$-Strahlung} % (fold)
\label{sub:reichweite_von_alpha_strahlung}

Neben elastischen St"o"sen und Ionisationsprozessen k"onnen $\alpha$-Teilchen ihre Energie auch durch Anregung oder Dissoziaion von Molek"ulen verlieren. Der Energieverlust pro Strecke ist dabei von der Energie der Strahlung und der Dichte des durchlaufenden Materials ab.

F"ur hinreichend gro"se Energien l"asst sich dies durch die Bethe-Bloch-Gleichung beschreiben, da bei niedrigen Energien Lad
ungsaustauschproze"se vermehrt auftauchen:

\begin{equation}
	-\frac{\mathrm{d}E_\mathrm{\alpha}}{\mathrm{d}x} = \frac{z^2e^4}{4\pi \epsilon_\mathrm{0} m_\mathrm{e}} \frac{n Z}{v^2} \ln \left( \frac{2 m_\mathrm{e}v^2}{I} \right)
\end{equation}

dabei ist $z$ die Ladung und $v$ die Geschwindigkeit der $\alpha$-Strahlung. $Z$ ist die Ordnungszahl, $n$ die Teilchendichte und $I$ die Ionisierungsenergie des Targetgases.

Die Reichtweite $R$ l"asst sich nun schreiben als:

\begin{equation}
	R = \int_0^E_\mathrm{0} \frac{\mathrm{d}E_\mathrm{\alpha}}{- \frac{\mathrm{d}E_\mathrm{\alpha}}{\mathrm{d}x}}
\end{equation}

Bei $\alpha$-Strahlung in Luft mit Energien von $E_\mathrm{\alpha} <= \SI{2.5}{\mega\electronvolt}$ gilt:

\begin{equation}
	R_\mathrm{m} = 3.1 \cdot E_\mathrm{\alpha}^\frac{3}{2}
\end{equation}

In Gasen bei konstaner Temperatur und konstantem Volumen ist die Reichweite von $\alpha$-Teilchen proportional zum Druck $p$. Daher kann durch variieren des Drucks $p$ eine Absorptionsmessung gemacht werden. Es gilt f"ur einen festen Abstand $x_\mathrm{0}$ zwischen Detektor und $\alpha$-Strahler:

\begin{equation}
	x = x_\mathrm{0} \frac{p}{\SI{1013}{\milli\bar}}
\end{equation}

\subsection{Halbleiter-Sperrschichtz"ahler} % (fold)
\label{sub:halbleiter_sperrschichtz_ahler}


\subsection{Diskreminatorschwelle} % (fold)
\label{sub:diskreminatorschwelle}
