\section{Diskussion}
\label{sec:diskussion}
	Die hier bestimmten mittleren Reichweiten 
	\begin{eqnarray*}
		R_\mathrm{m,1} & = & \SI{2.13(2)}{\centi \meter} \\	
		R_\mathrm{m,2} & = & \SI{2.14(16)}{\centi \meter}
	\end{eqnarray*}

	stimmen unter Ber"ucksichtigung des Fehlers "uberein.
	Das deutet auf eine realtiv gute Bestimmung dieses Wertes hin.
	F"ur eine belastbare Aussage wurden jedoch zu wenig Messungen durchgef"uhrt.

	Dies trifft auch auf die Werte der Energieabnahme $- \mathrm{d} E / \mathrm{d} x$ zu.
	Unter Ber"ucksichtigung der Fehler stimmen sie ebenfalls "uberein.

	Der Vergleich des Histogrammes mit der Gau"s- bzw. Poissonverteilung (siehe Abbildungen \ref{fig:verteilungGauss} und \ref{fig:verteilungPoisson}) passt in beiden F"allen recht gut.
	Der hier Ermittelte Wert in Bereich 6 f"allt lediglich aus dem Rahmen.
	Zudem ist festzuhalten, dass das Histogramm stark von der Wahl der Bereichsbreite $\Delta N$ abh"angt.
	Bei kleinerer Breite $\Delta N$ sind die einzelnen Werte des Histogramms in der N"ahe des Erwartungswertes starken Schwankungen unterzogen.
	Hier h"atte der Datenumfang wesentlich gr"o"ser sein m"ussen.

	Bei gr"o"seren Breiten $\Delta N$ veringert sich die Balkenzahl so sehr, dass sie nur noch schlecht mit einer Gau"skurve oder Poissonverteilung verglichen werden kann.

\begin{thebibliography}{9}
	\bibitem{anleitung} Physikalisches Anf"angerpraktikum der TU Dortmund: Versuch V701 - Reichweite von alpha-Strahlung. \url{http://129.217.224.2/HOMEPAGE/PHYSIKER/BACHELOR/AP/SKRIPT/V701.pdf}. Stand: Juni 2013.
\end{thebibliography}
