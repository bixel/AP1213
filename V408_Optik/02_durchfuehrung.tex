\section{Aufbau und Durchf"uhrung}
	\label{sec:durchfuehrung}

	Die Messapparatur befindet sich auf einer optischen Bank, auf denen die optischen Elemente verschoben werden k"onnen.
	Die Lichtquelle ist eine Halogenlampe und der Gegenstand ein "`Perl L"'.
	Vorweg wird die Gr"o"se von "`Perl L"' bestimmt.

	\subsection{Bestimmung der Brennweite durch Messung der Gegenstandsweite und Bildweite} % (fold)
		\label{sub:bestimmung_der_brennweite_durch_messung_der_gegenstandsweite_und_bildweite}
		
		Zun"achst wird die Brennweite von einer Linse mit bekannter Brennweite und einer mit unbekannter gemessen.
		Daf"ur wird auf die optische Bank eine Halogenlampe, der Gegenstand "`Perl L"' , eine Sammellinse unbekannter/bekannter Brennweite und ein Schirm angebracht.

		Nach Ausrichten der Linse mit gegebener Brennweite auf einer festen Gegenstandsweite $g$, wird der Schirm so eingestellt, dass das Bild scharf abgebildet wird.

		Das Wertepaar ($g_\mathrm{i}$, $b_\mathrm{i}$) wird f"ur insgesammt 10 verschiedene Gegenstandsweiten notiert.

		Anschlie"send wird dieselbe Messung mit einer Linse unbekannter Brennweite wiederholt.

	\subsection{Bestimmung der Brennweite einer Linse nach der Methode von Bessel} % (fold)
		\label{sub:bestimmung_der_brennweite_einer_linse_nach_der_methode_von_bessel}
		
		Der Abstand $e$ zwischen Schirm und Gegenstand wird konstant eingestellt, wobei dieser mindestens vier mal so gro"s wie die Brennweite $f$ der Linse sein sollte.

		Es werden zwei Linsenpositionen gesucht, an denen das Bild scharf abgebildet wird. Dabei handelt es sich um eine symmetrische Annordung und es gilt:

		\begin{equation}
			b_1 = g_2 \qquad \text{und} \qquad b_2 = g_1\, . \nonumber
		\end{equation}

		Wenn $g > b$ ist, wird das Bild verkleinert, andersrum vergr"o"sert.

		Es wird die Messung f"ur insgesammt 10 weitere Abst"ande $e_\mathrm{i}$ wiederholt.

		Die chromatische Abberation wird auf dieselbe Weise untersucht, nur dass hier jeweils ein blauer oder roter Filter vor den Gegenstand gesetzt wird.

		Diese Messung wird f"ur 5 verschiedene Abst"ande $e_\mathrm{i}$ wiederholt.

	\subsection{Bestimmung der Brennweite eines Linsensystems nach der Methode von Abbe} % (fold)
		\label{sub:bestimmung_der_brennweite_eines_linsensystems_nach_der_methode_von_abbe}

		Es handelt sich hierbei um ein 2-Linsen System, wobei diese wie eine dicke Linse angesehen werden.
		Dabei wird die Gegenstandsweite $g$ und die Bildweite $b$ jeweils zu den Hauptebenen $H$ und $H'$ gemessen, wie in Abb. \eqref{abbe} gezeigt.

		\begin{figure}[htbp]
			\centering
			\includegraphics[width = 12cm]{img/abbe.PNG}
			\caption{Bestimmund der Brennweite eines Linsensystems nach der Methode von Abbe}
			\label{abbe}
		\end{figure}

		Da die Positionen von $H$ und $H'$ nicht bekannt sind, wird sich ein beliebiger Referenzpunkt $A$ gesucht und zu diesem die Gegenstandsweite $g'$ und die Bildweite $b'$ gemessen.

		Es gilt:
		\begin{eqnarray}
			g' = g+h &=& f \cdot \left( 1 + \frac{1}{V} \right) + h \, , \label{abbe1}\\
			b' = b+h' &=& f \cdot (1 + V) + h' \, . \label{abbe2}
		\end{eqnarray}

		Die Linsen werden so dicht zusammengeschoben, dass diese sich ber"uhren und dieser Abstand wird beibehalten.

		Der Abbildungsma"sstab $V$ wird gemessen, sowie die Abst"ande $g'$ und $b'$ zum Referenzpunkt $A$.

		Diese Messung wird f"ur 10 Gegenstandsweiten $g'$ wiederholt.