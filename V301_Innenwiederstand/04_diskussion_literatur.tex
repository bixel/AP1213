\section{Diskussion}
\label{diskussion}
	Wie in Kapitel \ref{subsec:fehler} dargestellt, sind die bekannten Systematischen Fehler gering.
	Dennoch weichen die gemessenen Leistungswerte der Monozelle von der Theoriekurve ab (siehe Abb. \ref{fig:graph_monozelle_leistung}).
	Wie im Folgenden erl"autert wird, liegt das wom"oglich an einer fehlerhaften Messung der Spannungswerte $U_{0,-}$ mit ver"anderlichem Aussenwiderstand $R_\mathrm{A}$.

	Der Wert der direkten Spannungsmessung der Monozelle $U_{0,\mathrm{d}} = \SI{1.65 (2)}{\volt}$ und der Wert durch Ausgleichsrechnung bei Gegenspannung $U_{0,+} = \SI{1.68 (3)}{\volt}$ stimmen mit Ber"ucksichtigung der Fehler "uberein.
	Weil dieser Wert jedoch um etwa $\Delta U = \SI{.23}{\volt}$ vom Wert $U_{0,-}$ abweicht deutet dies ebenfalls auf eine fehlerhafte Messung mit variablem Widerstand $R_\mathrm{A}$ hin.

	Weil also zwei von drei Messungen auf das gleiche Ergebnis f"uhren, l"asst sich sagen, dass die Leerlaufspannung der Monozelle durch diesen Versuch zu

	\begin{equation*}
		(U_{0,\mathrm{d}} + U_{0,+}) / 2 = \SI{1.66 (4)}{\volt}
	\end{equation*}

	bestimmt werden kann. Dabei wird die erste, fehlerbehaftete Messung nicht ber"ucksichtigt.

	\enlargethispage{3cm}

\begin{thebibliography}{9}
	\bibitem{anleitung} Physikalisches Anf"angerpraktikum der TU Dortmund: Versuch Nr.301 - Leerlaufspannung und Innenwiderstand von Spannungsquellen. \url{http://129.217.224.2/HOMEPAGE/PHYSIKER/BACHELOR/AP/SKRIPT/V301.pdf}. Stand: Mai 2013.
\end{thebibliography}
