\section{Einleitung} % (fold)
\label{sec:einleitung}
	In diesem Versuch werden Funktionsweise und Kenngr"o"sen des Geiger-M"uller-Z"ahlrohrs untersucht.
	Das Ger"at erm"oglicht die Messung der Intensit"at ionisierender Strahlung.
	Auf Grund des einfachen Aufbaus ist das Geiger-M"uller-Z"ahlrohr kosteng"unstig und wegen seiner Verbreitung besonders interessant.

\section{Theorie} % (fold)
\label{sec:theorie}
	Zun"achst soll die Funktionsweise grob beschrieben werden.

	\subsection{Aufbau}
	\label{subsec:aufbau}
		Das Instrument besteht aus einem Anodendraht, der von einem Kathodenzylinder umschlossen ist.
		Der Raum zwischen Draht und Zylinder ist mit einem Gasgemisch niedrigend Drucks gef"ullt, das sich leicht ionisieren l"asst.
		Es wird eine Spannung $U$ zwischen $\SI{300}{\volt}$ und $\SI{2000}{\volt}$ an Anode und Kathode angelegt, wodurch ein radialsymmetrisches Feld im Innern des Zylinders entsteht.
		Der Zylinder ist von einem Stahlmantel umgeben, wobei eine Stirnseite aus einer d"unnen Membran aus Mylar besteht.
		Hierdurch wird m"oglichst wenig Strahlung beim Eintritt absorbiert und Gleichzeitig der Niederdruck im Inneren des Z"ahlrohrs bewahrt.

	\subsection{Funktionsweise}
	\label{subsec:funktionsweise}
		Wenn ein geladenes Teilchen in das Z"ahlrohr eintritt, gibt es seine Energie an die Gasatome ab und kann diese ioniseren, bis seine Energie aufgebraucht ist.
		Weil die Energie des einfallenden Teilchens wesentlich gr"o"ser ist, als die zur Ionisation ben"otigte Energie, ist die Anzahl ionisierter Kerne proportional zur Energie des Teilchens.
		Die freigesetzten Gas-Ionen werden nun durch das elektrische Feld abgelenkt und bei gen"ugend gro"ser Spannung $U$ in Anode und Kathode absorbiert.

		\subsubsection{Rekombination (I)}
		\label{subsubsec:rekombination}
			Bei zu geringer angelegter Spannung (beim vorliegenden Ger"at $U < \SI{300}{\volt}$) reicht die Feldst"arke im Zylinder nicht aus, um die Ionen vollst"andig zu trennen.
			Sie rekombinieren und die einfallende Strahlung l"asst sich nicht detektieren.

		\subsubsection{Ionisationskammer (II)}
		\label{subsubsec:ionisationskammer}
			Erh"oht man die Spannung, wird jedes ionisierte Molek"ul absorbiert und der Strom zwischen Anode und Kathode ist proportional zur Energie und zur Intensit"at der einfallenden Strahlung.
			Da der auftretende Strom jedoch sehr gering ist, kann nur Strahlung hoher Intensit"at gemessen werden.
			Man bezeichnet das Z"ahlrohr dann als Ionisationskammer.

		\subsubsection{Proportionalit"atsbereich (III)}
		\label{subsubsec:proportionalitaetsbereich}
			Bei gr"o"serer Spannung haben die im Zylinder freigestzten Elektronen gen"ugend Energie, um ihrerseits Molek"ule zu ionisieren.
			Auf diese Weise werden immer mehr Elektronen frei und man spricht man von einer \textsc{Townsend-Lawine}.


