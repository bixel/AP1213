\documentclass{scrartcl}
	%ngerman fur Umlaute, Anf"hrungszeichen, etc.%
	\usepackage{ngerman}

	%siunitx fur SI-Einheitesnsystem%
	\usepackage{siunitx}
	\sisetup{
	    locale=DE,
	    separate-uncertainty=true,
    	per-mode=fraction
	}

	%graphicx fur Bildeinbindung%
	\usepackage{graphicx}

	%UTF-8 encoding und richtige Silbentrennung
	%\usepackage[applemac]{inputenc}
	\usepackage[T1]{fontenc}
	\usepackage[utf8]{inputenc}
	\usepackage{lmodern}

	%enthaelt eqnref{}-Operator
	\usepackage{amsmath}

	%enthaelt pfeile mit text
	\usepackage{amssymb}

	%ermoeglicht Hyperlinks - Schein aber KACKE zu sein :D
	\usepackage{url}

	%damit lualatex formeln richtig darstellt
	%\usepackage{unicode-math}

	%Texteinzug vor Absatz entfernen%
	\parindent 0pt

\begin{document}
	
	\vspace*{3cm}

	\begin{center}
		\large
		TU Dortmund
	\end{center}

	\begin{center}
		\Huge
		V207 - Das Stefan-Boltzmann Gesetz
	\end{center}

	% \begin{center}
	% 	\large
	% 	Korrektur
	% \end{center}

	% 5cm fuer Korrektur, sonst 6cm vspace
	\vspace{6cm}
	\begin{center}
		\begin{minipage}[b]{8cm}
			\Large
			Markus Stabrin \\
			\normalsize
			markus.stabrin@tu-dortmund.de \\

			\Large
			Kevin Heinicke\\
			\normalsize
			kevin.heinicke@tu-dortmund.de \\
			\\
			\\

			Versuchsdatum: 4. Dezember 2012 \\
			\\
			Abgabedatum: 11. Dezember 2012
		\end{minipage}
	\end{center}

	\newpage

	\section{Einleitung} % (fold)
\label{sec:einleitung}

Bei der Erw"armung einer Metalloberfl"ache ist eine Elektronenemission m"oglich.
Dabei ist besonders die Austrittsarbeit von Bedeutung.
Dabei wird der Versuch im Hochvakuum durchgef"uhrt, damit keine Wechselwirkungen mit den Luftmolek"ulen stattfindet.

\section{Theorie} % (fold)
\label{sec:theorie}

\subsection{Austrittsarbeit und die Energieverteilung} % (fold)
\label{sub:austrittsarbeit_und_die_energieverteilung}

Metalle sind h"aufig kristalline Festk"orper.
Die Atome sind darin ionisiert und die Elektronen geh"oren nicht mehr zu einem bestimmten Atom sondern befinden sich im Kraftfeld s"amtlicher Ionen.
Darin k"onnen sich die Elektronen frei bewegen, wodurch eine hohe elektrische Leitf"ahigkeit erzeugt wird.
Um den Metallverband verlassen zu k"onnen, muss das Elektron gegen das Potential $\psi$ anlaufen k"onnen.
D"af"ur muss die Austrittsarbeit $e_\Mahrm{0}\psi$ geleistet werden.
Das Potential des Gitters kann in grober N"aherung als konstant betrachtet werden.
Abb. \ref{potential_topf} stellt das sogenannte Potentialtopfmodell dar.

\begin{figure}[!h]
	\centering
	\includegraphics[width = 5cm]{img/Potentialtopf.PNG}
	\caption{Potentialtopfmodell}
	\label{potential_topf}
\end{figure}

Die Elektronen des Kristallgitters unterliegen dem Pauli-Verbot, nach dem nur zwei Elektronen mit entgegengesetztem Spin denselben Zustand mit der Energie $E$ haben k"onnen.
Die Maximalenergie der Elektronen bei T = 0 wird als Grenzenergie $\psi$ bezeichnet.
Die Wahrscheinlichkeit daf"ur, ob ein Zustand mit der Energie $E$ besetzt wird, wird durch die Fermi-Dirac'sche Verteilungsfunktion angegeben:

\begin{equation}
	f(E) = \frac{1}{exp( \frac{E - \psi}{kT} ) + 1} \, .
\end{equation}

Der Verlauf ist in Abb. \ref{fermi} dargestellt.
Die Exponentialfunktion im Nenner "ubertrifft die Zahl 1 bei weitem, wodurch n"aherungsweise gilt:

\begin{equation}
	f(E) \propto exp ( \frac{\psi - E}{kT}) \, .
\end{equation}

\begin{figure}[!h]
	\centering
	\includegraphics[width = 5cm]{img/Fermi.PNG}
	\caption{Fermi-Dirac'sche Verteilungsfunktion}
	\label{fermi}
\end{figure}

\subsection{Berechnung der S"attigungsstromdichte bei der thermischen Elektronenemission} % (fold)
\label{sub:berechnung_der_s_attigungsstromdichte_bei_der_thermischen_elektronenemission}

F"ur die S"attigungsstromdichte $j_\mathrm{s}(T)$ erh"alt man nach Einf"uhrung eines kartesischen Koordinatensystems:

\begin{equation}
	j_\mathrm{s}(T) = 4\pi \frac{e_\mathrm{0} m_\mathrm{0} k^2}{h^3} T^2 exp( \frac{-e_\mathrm{0} \Phi}{kT}) \, .
\end{equation}

\subsection{Die Hochvakuum-Diode} % (fold)
\label{sub:die_hochvakuum_diode}

Um Wechselwirkungen mit den Gasmolek"ulen der Luft zu vermeiden, muss die Messung des S"attigungsstroms $j_\mathrm{s}$ im Hochvakuum duchgef"uhrt werden.
Diese ist nach Abb. \ref{diode} aufgebaut.
Durch eine angelegte Heizspannung kann die Gl"uhkathode auf 1000 bis 3000\,K erhitzt werden.

\begin{figure}[!h]
	\centering
	\includegraphics[width = 5cm]{img/Diode.PNG}
	\caption{Beschaltung einer Hochvakuum-Diode}
	\label{diode}
\end{figure}

	\section{Aufbau und Durchf"uhrung}
	\label{sec:durchfuehrung}

	Die Solarzelle wird gem"a"s Abb. \eqref{aufbau} angeschlossen.
	Die vier Solarzellen werden per Br"uckenstecker auf einer Rastersteckplatte mit einem  Amperemeter und einer Widerstandsdekade in Reihe geschaltet, parallel dazu ein Voltmeter.

	\begin{figure}[htbp]
		\centering
		\includegraphics[width = 12cm]{img/aufbau.PNG}
		\caption{Schaltskizze zur Messung der I-U Kennlinie \cite{anleitung}}
		\label{aufbau}
	\end{figure}

	"Uber der Solarzelle wird eine $\SI{120}{\watt}$ Lampe an einer h"ohenverstellbaren Vorrichtung angebracht.
	Bei "uberbr"uckter Widerstandsdekade wird die H"ohe so eingestellt, dass der Kurzschlussstrom $I_K$ $\SI{30}{\milli\ampere}$, $\SI{50}{\milli\ampere}$, $\SI{75}{\milli\ampere}$ und $\SI{100}{\milli\ampere}$ betr"agt.
	Mit dem gr"o"sten Abstand wird die Messung begonnen.
	Nach Unterbrechung des Stromkreises wird f"ur jede Messreihe die Leerlaufspannung $U_0$ gemessen.\\
	Um die Strom-Spannungs-Kennlinie bei den einzelnen H"ohen zu bestimmen, wird nun die Br"ucke entfernt und die Widerstandsdekade von $\SI{1}{\ohm}$ bis $\SI{250}{\ohm}$ variiert.
	Dabei wird nach jeder Messung die Leistung mit Hilfe von $P = U \cdot I$ errechnet, um m"oglichst zeitnah die Messpunkte um das Leistungsmaximum durchzuf"uhren.
	Dabei ist darauf zu achten die Messreihen m"oglichst z"ugig aufzunehmen, da die Solarzelle sich aufheitzt.\\
	Aus den erhaltenen Daten wird nun der Wirkungsgrad bestimmt.

	\section{Auswertung}
\label{sec:auswertung}

\subsection{Mittlere Wegl"ange} % (fold)
\label{sub:mittlere_wegl_ange}


\begin{table}[!h]
\begin{center}
\begin{tabular}{|r|r|r|r|}
\hline
T[$\SI{}{^\circ}$] & $p_\mathrm{s"at}[\SI{}{\milli\bar}]$ & $\bar{w}[\SI{}{\centi\meter}]$ & $\frac{a}{\bar{w}}$\\
\hline
\hline
25  & 0.0052  & 0.5532 & 1.81\\
105 & 0.6924  & 0.0042 & 238.75\\
150 & 4.7948  & 0.0006 & 1653.38\\
190 & 19.5284 & 0.0001 & 6733.94\\
\hline
\end{tabular}
\caption[]{Errechnete Werte f"ur $p_\mathrm{s"at}$ und $\bar{w}$ in Abh"angigkeit von der Temperatur.}
\label{tab:weg}
\end{center}
\end{table}

Das Verh"altnis von der L"ange der verwendeten R"ohre $a$ zu der freien Wegl"ange der Elektronen $\bar{w}$ soll etwa einen Faktor von $1000 - 4000$ ergeben.
Nach den Formeln \eqref{p} und\eqref{w} ergaben sich die Werte aus Tabelle \ref{tab:weg} f"ur $p_\mathrm{s"at}$ und $\bar{w}$ mit $a = \SI{1}{\centi\meter}$.

Es ist zu erkennen, dass die Temperaturen $\SI{25}{^\circ}$ und $\SI{105}{^\circ}$ nicht f"ur eine Franck-Hertz Kurve geeignet sind, da das Verh"altnis zu niedrig ist.
Bei der Temperatur von $\SI{190}{^\circ}$ ist es schon etwas zu hoch, jedoch lassen sich noch akzeptable Ergebnisse erzielen, w"ahrend bei $\SI{150}{^\circ}$ das Verh"altnis sehr gut ist.

\subsection{Energieverteilung der beschleunigten Elektronen} % (fold)
\label{sub:energieverteilung_der_beschleunigten_elektronen}

\subsubsection{$\si{25}{^\circ C}$} % (fold)
\label{sub:_si}


\begin{table}[!h]
\begin{center}
\begin{tabular}{|r|r|}
\hline
$U_\mathrm{a}$[V] & $\Delta y$ \\
\hline
\hline

7.000 	& 0.1 \\
7.125 	& 0.1 \\
7.250 	& 0.1 \\
7.375 	& 0.1 \\
7.500 	& 0.1 \\
7.625 	& 0.1 \\
7.750 	& 0.1 \\
7.875 	& 0.1 \\
8.000 	& 0.1 \\
8.125 	& 0.1 \\
8.250 	& 0.2 \\
8.375 	& 0.2 \\
8.500 	& 0.3 \\
8.625 	& 0.3 \\
8.750 	& 0.4 \\
8.875 	& 0.3 \\
9.000 	& 0.4 \\
9.125   & 0.4 \\
9.250 	& 0.6 \\
9.375 	& 0.8 \\
9.500 	& 1.2 \\
9.625 	& 2.0 \\
9.750 	& 3.8 \\
9.875 	& 1.5 \\

\hline
\end{tabular}
\caption[]{Wertepaare der Energieverteilung bei $\SI{25}{^\circ}$.}
\label{tab:a1}
\end{center}
\end{table}

\begin{figure}[!h]
	\centering
	\includegraphics[width = 10cm]{img/t20.pdf}
	\caption{Energieverteilung der beschleunigten Elektronen bei $\SI{25}{^\circ}$.}
	\label{gra}
\end{figure}

Bei der Messung der Energieverteilung der beschleunigten Elektronen bei Zimmertemperatur ergab sich ein maximaler Anstieg der Kurve bei:

\begin{equation}
	U_\mathrm{a,max} = \SI{9.75}{\volt}
\end{equation}

Mithilfe der Formel:

\begin{equation}
	U_\mathrm{b} - U_\mathrm{a,max} = K
\end{equation}

ergibt sich f"ur $U_\mathrm{b} = \SI{11}{\volt}$:

\begin{equation}
	K = \SI{1.25}{\volt}
\end{equation}

Qualitativ l"asst sich sagen, dass der gr"o"ste Teil der Elektronen Energien von $\SI{9.25}{\volt} - \SI{10}{\volt}$ besitzt.

\subsubsection{$\si{150}{^\circ C}$} % (fold)
\label{sub:_si}


\begin{table}[!h]
\begin{center}
\begin{tabular}{|r|r|}
\hline
$U_\mathrm{a}$[V] & $\Delta y$ \\
\hline
\hline

1.00 &	0.8  \\
1.25 &	0.8  \\
1.50 &	0.8  \\
1.75 &	0.8  \\
2.00 &	0.9  \\
2.25 &	0.8  \\
2.50 &	1.0  \\
2.75 &	0.8  \\
3.00 &	0.8  \\
3.25 &	0.9  \\
3.50 &	0.8  \\
3.75 &	0.8  \\
4.00 &	0.5  \\
4.25 &	0.4  \\
4.50 &	0.4  \\
4.75 &	0.1  \\
5.00 &	0.0  \\
5.25 &	0.0  \\
5.50 &	0.0  \\
5.75 &	0.0  \\

\hline
\end{tabular}
\caption[]{Wertepaare der Energieverteilung bei $\SI{150}{^\circ}$.}
\label{tab:a1}
\end{center}
\end{table}

\begin{figure}[!h]
	\centering
	\includegraphics[width = 10cm]{img/t150.pdf}
	\caption{Energieverteilung der beschleunigten Elektronen bei $\SI{150}{^\circ}$.}
	\label{gra}
\end{figure}


	\section{Diskussion}
\label{diskussion}

	

\begin{thebibliography}{9}
	\bibitem{anleitung} Physikalisches Anf"angerpraktikum der TU Dortmund: Versuch V351 - Fourier-Analyse und Synthese. \url{http://129.217.224.2/HOMEPAGE/PHYSIKER/BACHELOR/AP/SKRIPT/V351.pdf}. Stand: Mai 2013.
\end{thebibliography}


\end{document}