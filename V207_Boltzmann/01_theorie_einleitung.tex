\section{Einleitung}
	\label{sec:einleitung}
	In diesem Versuch wird mit Hilfe des Stefan-Boltzmann Gesetzes das Absorptions- und Emissionsverm"ogen verschiedener Oberfl"achen untersucht.

\section{Theorie}
	\label{sec:theorie}
	Jeder K"orper strahlt W"arme in Form von infraroter Strahlung ab oder absorbiert sie.
	Wie gut die W"arme aufgenommen oder abgegeben werden kann wird durch das E\-mis\-sions\-ver\-m"o\-gen $\epsilon$, beziehungsweise das Absorptionsverm"ogen $A$ beschrieben.
	Die Gr"o"sen $\epsilon$ und $A$ nehmen dabei Werte zwischen 0 und 1 an und h"angen haupts"achlich von der Temperatur $T$ des K"orpers und der Wellenl"ange $\lambda$ der Strahlung ab.
	Ein Wert $\epsilon = A = 1$, bedeutet, dass die gesamte Energie abgestrahlt wird.
	Man bezeichnet den K"orper dann als Wei"sen K"orper.
	Bei einem Wert 0 strahlt der K"orper keine Energie ab und er hei"st Schwarzer K"orper.
	Eingestrahlte Energie wird dementsprechend bei $A = 1$ v"ollig absorbiert und bei $A = 0$ reflektiert.
	Es gilt

	\begin{equation*}
		\epsilon (\lambda, T) = A(\lambda, T) = 1 - R(\lambda, T) .
	\end{equation*}

	Dabei bezeichnet $R$ das Reflexionsverm"ogen.


	Die Extremf"alle $\epsilon = 0$ und $\epsilon = 1$ treten in der Realit"at jedoch nicht auf.
	Man erreicht jedoch ann"ahernd Schwarze K"orper, indem man einen Hohlraum mit nur einer kleinen "Offnung benutzt.
	Strahlung kann hier eintreten und wird im Inneren mehrfach reflektiert, wobei jedes Mal ein Teil der Energie absorbiert wird.

	Alle K"orper, die ein Emissionsverm"ogen < 1 besitzen nennt man Graue K"orper.

	\subsection{Planck'sches Strahlungsgesetz}
		\label{sub:gesetze}

		Die von einem K"orper abgestrahlte Leistung $P$, die sich "uber einen bestimmten Raum\-win\-kel $\Omega_0$ verteilt,
		l"asst sich durch das Planck'sche Strahlungsgesetz beschreiben:

		\begin{equation*}
			P(\lambda, T) = \frac{2 \pi c^2 \hbar}{\Omega_0 \lambda^5} \left[ \exp \left(\frac{c \hbar}{k \lambda T} - 1\right) \right] ^{-1} .
		\end{equation*}

		Hierbei ist $k$ die Boltzmann-Konstante, $c$ die Lichtgeschwindigkeit und $\hbar$ die Planck-Konstante.

		Es l"asst sich erkennen, dass sich das Maximum der Leistung f"ur h"ohere Temperaturen $T$ zu geringen Wellenl"angen $\lambda$ verschiebt und immer spitzer wird.
		Durch numerische L"osung der Gleichung

		\begin{equation*}
			\frac{\partial P(\lambda, T)}{\partial \lambda} = 0
		\end{equation*}

		folgt f"ur die Wellenl"ange $\lambda_\mathrm{max}$ der maximal abgestrahlten Leistung $P_\mathrm{max}$ bei einer be\-stim\-mten Temperatur $T$:

		\begin{equation*}
			\lambda_\mathrm{max} T = \SI{2897.8}{\micro \meter \kelvin} .
		\end{equation*}


	\subsection{Stefan-Boltzmann Gesetz}

		Integriert man $P(\lambda, T)$ "uber alle Wellenl"angen, erh"alt man das Stefan-Boltzmann Gesetz.
		Es beschreibt die gesamte abgestrahlte Leistung in Abh"angigkeit der Temperatur:

		\begin{equation*}
			\label{boltzmann}
			P(T) = \epsilon \sigma T^4 .
		\end{equation*}

		Wobei die Stefan-Boltzmann Konstante $\sigma = \SI{5.67e-8}{\watt \per \meter \squared \per \raiseto{4} \kelvin}$ nicht zu verwechseln ist mit der Boltzmann-Konstante $k$.