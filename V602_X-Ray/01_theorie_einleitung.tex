\section{Einleitung}
	Um die "au"seren H"ullenelektronen zu untersuchen reicht es diese mit UV- und IR-Strahlung zu untersuchen. Zugang zu tieferliegenden Teilen der Atomh"ulle bekommt man jedoch nur mit der energiereichen R"ontgenstrahlung, dessen Energiebereich sich von etwa $\SI{10}{\elektron\volt}$ bis $\SI{100}{\elektron\volt}$ erstreckt.
	\vspace{0.3cm}
	Zun"achst wird die Erzeugung der R"ontgenstrahlung und das resultierende Emissionsspektrum beschrieben, um anschlie"send die Wechselwirkung mit Materie mit ihren Absorptionsspektren zu betrachten. Im letzten Teil interessieren uns Verfahren zur Energie- und Intensit"atsmessung bei R"ontgenstrahlung.

\section{Funktionsweise und Theoretische Grundlagen}
	